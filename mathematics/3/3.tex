% UTF-8
% pt-BR

\begin{question}
  Considere o paralelepípedo ilustrado ao lado.
  \begin{enumerate}
    \item Escreva o segmento de reta orientado $\overrightarrow{AJ}$ como uma combinação linear dos vetores $\vec u_1$, $\vec u_2$ e/ou $\vec u_3$, sabendo que o comprimento do segmento $\overline{CJ}$ é o dobro de $\overline{JG}$.
    \item Escreva o ponto $I$ como a soma do ponto $H$ com uma combinação linear dos vetores $\vec u_1$, $\vec u_2$ e/ou $\vec u_3$, sabendo que $I$ é o ponto médio do segmento $\overline{BF}$.
  \end{enumerate}

  \begin{center}
    \begin{tikzpicture}[scale=0.8]

	\newcommand\angleA{-5}
	\newcommand\angleB{+30}
	\newcommand\angleC{+70}
	\newcommand\sideA{3}
	\newcommand\sideB{2}
	\newcommand\sideC{3}

	\clip (-0.5,-1) rectangle (7,5);
	
	\draw [->, very thick] (0,0) -- (\angleA:\sideA);
	\draw[dashed, very thick] [->] (0,0) -- (\angleB:\sideB);
	\draw [->, very thick] (0,0) -- (\angleC:\sideC);
	\draw [-] (\angleC:\sideC) -- ++(\angleB:\sideB) -- ++(\angleA:\sideA);
	\draw [-] (\angleC:\sideC) -- ++(\angleA:\sideA) -- ++(\angleB:\sideB);
	\draw[dashed] [-] (\angleB:\sideB) -- ++(\angleC:\sideC);
	\draw[dashed] [-] (\angleB:\sideB) -- ++(\angleA:\sideA);
	\draw (0,0) ++(\angleA:\sideA) ++(\angleB:\sideB) -- ++(\angleC:\sideC);
	\draw (0,0) ++(\angleA:\sideA) -- ++(\angleC:\sideC);
	\draw (0,0) ++(\angleA:\sideA) -- ++(\angleB:\sideB);

	\draw (0,0) node[anchor=north east] {$A$};
	\draw (\angleA:\sideA) node[anchor=north west] {$B$};
	\draw (\angleA:\sideA) ++(\angleB:\sideB) node[anchor=north west] {$C$};
	\draw (\angleB:\sideB) node[anchor=south east] {$D$};
	\draw   (\angleC:\sideC) node[anchor=south east] {$E$}
	      ++(\angleA:\sideA) node[anchor=south] {$F$}
	      ++(\angleB:\sideB) node[anchor=south west] {$G$};
	\draw (\angleC:\sideC) ++(\angleB:\sideB) node[anchor=south east] {$H$};
	\draw (\angleA:\sideA) ++(\angleC:\sideC/2) node[anchor=west] {$I$};
	\draw (\angleA:\sideA) ++(\angleB:\sideB) ++(\angleC:2/3*\sideC) node[anchor=west] {$J$};
	\draw[fill=black] (0,0) ++ (\angleA:\sideA) ++(\angleC:\sideC/2) circle (0.05);
	\draw[fill=black] (0,0) ++ (\angleA:\sideA) ++(\angleB:\sideB) ++(\angleC:2/3*\sideC) circle (0.05);

	\draw (0,0) ++(\angleA:\sideA/2) node[anchor=north] {$\vec u_1$};
	\draw (0,0) ++(\angleB:3/4*\sideB) node[anchor=north west] {$\vec u_2$};
	\draw (0,0) ++(\angleC:\sideC/2) node[anchor=east] {$\vec u_3$};

\end{tikzpicture}
  \end{center}

  %-------------------
  \begin{answer}
    \begin{enumerate}
        \item $\overrightarrow{AJ} = \vec u_1 + \vec u_2 + \frac{2}{3}\vec u_3$.
        \item $I = H + \vec u_1 - \vec u_2 - \frac{1}{2} \vec u_3$.
    \end{enumerate}
  \end{answer}

  %-------------------
  \begin{solution}
    \begin{enumerate}
      \item $\overrightarrow{AJ} = \overrightarrow{AB} + \overrightarrow{BC} + \overrightarrow{CJ}
                                 = \vec u_1 + \vec u_2 + \frac{2}{3}\vec u_3$.

      \item
      $
        I = H + \overrightarrow{HG} + \overrightarrow{GF} + \overrightarrow{FI}
          = H + \vec u_1 + \left(-\vec u_2\right) + \left(-\frac{1}{2}\vec u_3\right)
          = H + \vec u_1 - \vec u_2 - \frac{1}{2} \vec u_3
      $.
    \end{enumerate}
  \end{solution}
\end{question}
