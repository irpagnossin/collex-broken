% UTF-8
% pt-BR

\begin{question}
  Proque que, se $\vec u + \vec v = \vec w$, então $\vec v = \vec w - \vec u$.
  Faça isso utilizando apenas as quatro propriedades da soma de vetores.

  \begin{solution}
    \begin{align*}
      \vec u + \vec v &= \vec w &\\
      -\vec u + \vec u + \vec v &= -\vec u + \vec w &\text{Soma o elemento oposto}\\
      \vec u + \left(-\vec u\right) + \vec v &= \vec w + \left(-\vec u\right) &\text{Prop. comutativa}\\
      \left[\vec u + \left(-\vec u\right)\right] + \vec v &= \vec w + \left(-\vec u\right) &\text{Prop. associativa}\\
      \left[\vec 0\right] + \vec v &= w - \vec u &\text{Definição da subtração}\\
      \vec v &= \vec w - \vec u &\text{Elemento neutro da soma}
    \end{align*}
  \end{solution}
\end{question}
