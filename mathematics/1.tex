% UTF-8
% pt-BR
\tikzset{>=stealth',every on chain/.append style={join},every join/.style={->}}

\begin{question}
	Identifique se o conjunto $\{\vec u + \vec v, \vec u - \vec v\}$ é linearmente dependente (LD) ou independente (LI), sabendo que $\{\vec u, \vec v\}$ é LI.
	Explique sua conclusão.

	\begin{answer}
		O conjunto é LI.
	\end{answer}

	\begin{solution}
		Para determinar se $\{\vec u + \vec v, \vec u - \vec v\}$ é linearmente dependente (LD) ou independente (LI), precisamos identificar quais combinações lineares desses vetores resultam no vetor nulo.
		Isto é, precisamos analisar a solução da equação
		\begin{equation*}
			\alpha \left(\vec u + \vec v\right) + \beta \left(\vec u - \vec v\right) = \vec 0,
			\quad \alpha, \beta \in \mathbb{R}.
		\end{equation*}

		Se essa combinação linear só for possível para $\alpha = \beta = 0$ (combinação linear trivial), o conjunto será LI; caso contrário, será LD.

		Então vejamos:
		\begin{equation*}
			\alpha (\vec u + \vec v) + \beta (\vec u - \vec v) = \vec 0 \Rightarrow
			\alpha\vec u + \alpha\vec v + \beta\vec u - \beta\vec v = \vec 0 \Rightarrow
			(\alpha+\beta)\vec u + (\alpha-\beta)\vec v = \vec 0.
		\end{equation*}

		Mas $\{\vec u,\vec v\}$ é LI, de onde podemos afirmar que $\alpha + \beta = 0$ e $\alpha - \beta = 0$.
		Esse sistema de equações só pode ser resolvido se $\alpha = \beta = 0$.
		Logo, apenas a combinação linear trivial desses vetores resulta no vetor nulo e, por conseguinte, $\{\vec u + \vec v, \vec u - \vec v\}$ é LI.
	\end{solution}
\end{question}

