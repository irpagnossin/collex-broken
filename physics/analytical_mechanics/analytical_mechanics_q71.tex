\begin{question}
    Considere uma partícula de massa \measure{0.500}{kg} movendo-se num plano ao longo da trajetória dada por $q^1 \equiv r(t) = 2$ e $q^2 \equiv \theta(t) = 4t$, onde $(q^1, q^2) \equiv (r,\theta)$ é um sistema de coordenadas polar e $t$ representa o tempo (em segundos).
    Determine a energia cinética dessa partícula, $T := \frac{1}{2} m |\vec v|^2 = \frac{1}{2} m v^i v_i = \frac{1}{2} m g_{ij} v^i v^j$.
    Considere ainda que $(ds)^2 = (dq^1)^2 + (q^1\, dq^2)^2$ tenha sido construído de modo que $ds$ seja dado em metros, de sorte que a velocidade é dada \unit{m/s} (consequentemente, a energia cinética é dada em joules).

    \begin{answer}
      16,0 ou \measure{16.0}{J}
    \end{answer}

    \begin{solution}
      A trajetória seguida pelo objeto é aquela descrita na questão anterior, de modo que podemos emprestar dela a velocidade: $|v| = \measure{8}{m/s}$. Logo, $T = \half \cdot 0.5 \cdot 8^2 = \measure{16}{J}$.
    \end{solution}
\end{question}