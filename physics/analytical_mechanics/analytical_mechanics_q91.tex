\begin{question}
    Considere novamente o cenário da conta presa a um arame parabólico e puxada por um fio.
    Determine a equação de movimento da conta utilizando a ``forma 2'' da equação de Lagrange.
    \begin{enumerate}
      \item $4ma^2x\dot x^2 + m(1 + 4a^2 x^2)\ddot x = 2ax(f_z - mg) + f_x$ \rightanswer
      \item $m\ddot x(1 + 4a^2x^2) + 4ma^2 x \dot x^2 + 2mgax = 0$
      \item $m\ddot x(1 + 4a^2x^2) + 4ma^2 x \dot x^2 + 2mgax = 2ax(f_z - mg) + f_x$
      \item $m\ddot x(1 + 4a^2x^2) + 4ma^2 x \dot x^2 + 2mgax = f_x$
    \end{enumerate}

    \bigskip
    \begin{compactdesc}
      \item[Observação:] note que se fizermos $f_x = f_z = 0$ (\ie, se removermos o fio), obteremos a equação de movimento do caso anterior, sem fio.
    \end{compactdesc}

    \begin{solution}
      O lado esquerdo da equação de Euler-Lagrange foi determinado na questão \ref{q:eq-movimento} e o direito, na questão \ref{q:Fx}.
      Basta juntar as duas para obter a equação de movimento: $4ma^2x\dot x^2 + m(1 + 4a^2 x^2)\ddot x = 2ax(f_z - mg) + f_x$.
    \end{solution}
\end{question}