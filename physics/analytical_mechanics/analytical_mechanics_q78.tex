\begin{question}
    Considere os dois objetos ilustrados abaixo, conectados por uma mola e livres para mover no plano $xy$.
    Quantos são os graus de liberdade?

    \begin{center}
      \includegraphics[width=0.4\textwidth]{20180719_193333}
    \end{center}

    \begin{enumerate}
      \item 0
      \item 1
      \item 2
      \item 3
      \item 4 \rightanswer
    \end{enumerate}

    \begin{solution}
      Imagine o que podemos fazer com esse sistema: podemos movê-lo horizontalmente (\eg, coordenada $x$ do centro de massa) e verticalmente (coordenada $y$), podemos ainda girá-lo em torno do centro (coordenada $\theta$, digamos) e, finalmente, podemos alterar a distância entre os objetos, sem mover o centro de massa (coordenada $d$).
      Então, $(x,y,\theta,d)$ é um conjunto de quatro coordenadas generalizadas que é capaz de descrever a configuração desse sistema.
      Portanto, há quatro graus de liberdade.

      Outra estratégia é a seguinte: como cada objeto tem $N = 2$ graus de liberdade (pois estão no plano $xy$), o \emph{sistema} tem $2N = 4$ graus de liberdade.
      Note que, embora haja uma mola conectando os dois objetos, ela não impõe uma restrição à distância entre os dois objetos.
      Ou seja, não há equações de vínculo.
      Note que se houvesse uma barra rígida ligando os dois objetos, ao invés da mola, teríamos uma equação de vínculo e, por isso, um grau de liberdade a menos.
    \end{solution}
\end{question}