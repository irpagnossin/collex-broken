\begin{question}
    Ainda sobre o cenário da conta presa a um arame parabólico, qual é a expressão da lagrangeana, escrita em termos do único grau de liberdade?
    \begin{enumerate}
      \item $L = \half m \dot x^2 (1 + 4a^2 x^2) - mg(ax)^2$ \rightanswer
      \item $L = \half m \dot x^2 (1 + a^2) - mgz$
      \item $L = \half m \dot x^2 (1 + a^2) - mg(ax)^2$
      \item $L = \half m \dot x^2 (1 + 4a^2 x^2)$
      \item $L = \half m \dot x^2 (1 + 4a^2 x^2) + mg(ax)^2$
    \end{enumerate}

    \begin{solution}
      Utilizando as expressões de $T$ e $U$, obtidos nas duas questões anteriores, na definição da função lagrangeana, $L := T - U$, obtemos a resposta: $L = \half m \dot x^2 (1 + 4a^2 x^2) - mg(ax)^2$.
    \end{solution}
\end{question}