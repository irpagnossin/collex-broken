\begin{question}
    Considere uma partícula movendo-se num plano ao longo da trajetória dada por $q^1 \equiv r(t) = 3t$ e $q^2 \equiv \theta(t) = \pi/2$, onde $(q^1, q^2) \equiv (r,\theta)$ é um sistema de coordenadas polar e $t$ representa o tempo.
    Determine as componentes contra-variantes do vetor de velocidade $v^i := dq^i/dt \equiv \dot q^i$.
    Em outras palavras, determine $v^1 = dq^1/dt$ e $v^2 = dq^2/dt$.
    \begin{enumerate}
      \item $v^1 = 3$ \rightanswer
      \item $v^2 = 0$ \rightanswer
      \item $v_1 = 0$
      \item $v_2 = 1$
      \item $v^1 = 0$
    \end{enumerate}

    Note que o cenário descrito nesta questão é o de uma partícula que afasta-se da origem do sistema de referência com velocidade radial constante de \measure{3}{m/s}.

    \begin{solution}
      $q^1 = 3t \Rightarrow v^1 \equiv dq^1/dt \equiv \dot q^1 = 3$ e $q^2 = \pi/2 \Rightarrow v^2 \equiv dq^2/dt \equiv \dot q^2 = 0$.
      Nesse contexto, $t$ tem um papel bastante similar ao $\tau$, usado no cálculo do comprimento de arcos.
      A diferença é que $\tau$ é qualquer parâmetro que possa ser usado para descrever a curva, enquanto $t$ é um parâmetro específico: o tempo.
    \end{solution}
\end{question}