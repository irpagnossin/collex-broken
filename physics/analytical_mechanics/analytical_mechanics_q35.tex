\begin{question}
    Determine o momento angular orbital da Terra, sabendo que sua massa é de \measure{5.97e24}{kg} e que a distância média até o Sol é de \measure{1.50e9}{m} (considere que sua órbita é circular).

    \begin{answer}
      \measure{2.68e36}{kg.m^2/s}
    \end{answer}
    
    \begin{solution}
      Considerando que a órbita seja circular, os vetores de posição e de velocidade são perpendiculares.
      Então, $L = mrv$, onde $m$ é a massa da Terra, $r$ é o raio da órbita e $v$ é a velocidade.
      Mas $v = 2\pi r/T$, onde $T = \measure{31449860}{s}$ (duração do ano).
      Então, $L = \measure{2.68e36}{kg.m^2/s}$.
    \end{solution}
\end{question}