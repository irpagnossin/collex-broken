\begin{question}
  	Existe uma ``forma 3'' da equação de Lagrange, muito útil quando nosso objetivo é determinar as forças de vínculo:
  	\begin{equation*}
  		\frac{d}{dt}\pd{L}{\dot q^i} - \pd{L}{q^i} = \sum_{j = 1}^{n} \lambda_i \pd{f_j}{q^i},
  	\end{equation*}
  	onde $f_j$ são as $n$ equações de vínculo na forma $f_j(q^i) = \text{constante}$.
  	$\lambda_i$ são os \emph{multiplicadores de Lagrange}, um para cada equação de vínculo (note que a somatória ocorre nas equações de vínculo, de modo que se houver $n$ delas, haverá $n$ parcelas do tipo $\lambda_i \pd{f_j}{q^i}$).

  	O procedimento é o seguinte:
  	\begin{compactenum}[(i)]
  		\item Utilize a ``forma 1'' ou ``forma 2'', em termos apenas das variáveis livres, para determinar as equações de movimento, como fizemos até agora.
  		\item Escreva as equações de vínculo na forma $f_j(q^i) = \text{constante}$.
  		\item Utilize a ``forma 3'' (ou ``forma 4'', abaixo) em termos de todas as variáveis do sistema (\ie, sem aplicar as equações de vínculo) para determinar as equações de movimento.
  		\item Utilize as soluções do primeiro passo nas equações do passo anterior para determinar os $\lambda_j$.
  		\item Identifique $Q_i = \sum_{j = 1}^{n} \lambda_i \pd{f_j}{q^i}$ como a força generalizada de vínculo conjugada à variável $q^i$.
  	\end{compactenum}

  	Vamos aplicar esse procedimento no caso da questão \ref{q:atwood} (máquina de Atwood sem rotação).
  	O primeiro passo já foi feito naquela questão.
  	O segundo passo é escrever as equações de vínculo na forma $f_j(q^i) = \text{constante}$.
  	Escolha a alternativa que representa esse resultado.

  	\begin{enumerate}
  		\item $f_1(z_1, z_ 2) = z_1 + z_2$ \rightanswer
  		\item $f_1(z_1, z_ 2) = C - z_2$
  		\item $f_1(z_1, z_ 2) = C - z_1$
  		\item $f_1(z_1, z_ 2) = C$
  		\item $f_1(z_1, z_ 2) = z_1 - z_2$
  	\end{enumerate}

  	\bigskip
  	\begin{compactdesc}
  		\item[Observação:] caso você esteja trabalhando com a ``forma 2'' da equação de Euler-Lagrante (aquela que usa $T$ ao invés de $L$), é possível escrever uma ``forma 4'', assim:
  		\begin{equation*}
  			\frac{d}{dt}\pd{T}{\dot q^i} - \pd{T}{q^i} = F_i + \sum_{j = 1}^{n} \lambda_i \pd{f_j}{q^i}.
  		\end{equation*}
  	\end{compactdesc}

    \begin{solution}
      Na máquina de Atwood, o comprimento do cabo que liga $m_1$ com $m_2$ não muda.
      Por isso, $z_1 + z_2 = C$, onde $C$ é uma constante.
      Essa equação já está na forma $f(z_1, z_2) = C$, com $f(z_1, z_2) = z_1 + z_2$.
    \end{solution}
\end{question}