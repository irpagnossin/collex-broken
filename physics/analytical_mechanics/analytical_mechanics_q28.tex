\begin{question}
    Determine o momento de inércia de um aro (um anel fino) de raio \measure{2.00}{m} e massa \measure{0.500}{kg}.
    \begin{answer}
      \measure{2.00}{kg.m^2}
    \end{answer}

    \begin{solution}
      Considere o elemento de comprimento $dl$ desse aro, ao qual podemos associar um elemento de massa $dm = \lambda\, dl$, onde $\lambda = m /(2\pi r)$ é a densidade linear de massa do aro, $m$ é sua massa e $r$, seu raio.
      Todos os elementos $dm$ desse aro estão à distância $r$ do eixo de interesse, que passa pelo centro, de modo que o elemento de momento de inércia é simplesmente $dI = r^2\,dm$.
      Assim,
      \begin{equation*}
        dI = r^2\, dm
          \Rightarrow
        \int dI = r^2 \int dm
          \Rightarrow
        I = r^2 \lambda \int_0^{2\pi r} dl
          = r^2 \lambda 2 \pi r
          = m r^2,
      \end{equation*}
      onde usamos $\lambda = m /(2\pi r)$ na última passagem. Agora, usando $r = \measure{2.00}{m}$ e $m = \measure{0.500}{kg}$, obtemos $I = \measure{2.00}{kg.m^2}$
    \end{solution}
\end{question}