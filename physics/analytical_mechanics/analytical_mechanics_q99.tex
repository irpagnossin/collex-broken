\begin{question}
  	Imagine que conectemos uma das extremidades de uma mola no chão, abaixo de uma mesa, e a outra extremidade a um fio, que por sua vez passa por um furo na mesa e está conectado a uma pequena bolinha de massa $m$ (veja a figura).
  	O arranjo é tal que, quando $r = 0$, a mola, cuja constante elástica é $k$, não está nem esticada nem comprimida.
    Determine as equações de movimento dessa bolinha em termos das variáveis $(r, \theta)$.

    \begin{center}
      \includegraphics[width=0.4\textwidth]{20180718_231449}
    \end{center}

    \begin{enumerate}
      \item $m\ddot r - mr\dot\theta^2 = -kr$ \rightanswer
      \item $2mr\dot r\dot\theta + mr^2\ddot \theta = 0$ \rightanswer
      \item $L = \frac{1}{2}m\left(\dot r^2 + r^2\dot\theta^2\right) - \frac{1}{2}kr^2$
      \item $m\ddot r - m\dot r - mr\dot\theta^2 + kr = 0$
      \item $2mr\dot r\dot\theta + mr^2\ddot \theta  - mr^2\dot\theta= 0$
    \end{enumerate}

    \bigskip
    \begin{compactdesc}
    	\item[Observação:] este é um exemplo de movimento sob a ação de uma força central.
    	\item[Observação:] este problema ilustra o fato de que sempre existe uma \emph{constante do movimento} associada à variável $q^i$ quando $\pd{L}{q^i} = 0$ para essa variável.
    	Nesse caso, $\pd{L}{\theta} = 0$.
    	Como consequência, $\frac{d}{dt}\left(mr^2\dot \theta\right) = 0$, ou seja, a quantidade $mr^2\dot \theta$ (momento angular) é constante; é, portanto, uma constante do movimento.
    \end{compactdesc}

    \begin{solution}
      Primeiramente, precisamos determinar a função lagrangeana $L := T - U$.
      Utilizando o sistema de coordenadas ilustrado, a energia cinética pode ser escrita diretamente como $T = \half m (\dot r^2 + r^2 \dot\theta^2)$.
      A energia potencial, por outro lado, é totalmente devida à mola (a gravidade não afeta o movimento, pois ele ocorre no plano horizontal): $U = \half k r^2$, onde usamos o fato de que $r = 0$ é o comprimento natural da mola.
      Assim, $L = \half m (\dot r^2 + r^2 \dot\theta^2) - \half k r^2$.
      Em seguida determinamos as derivadas pertinentes:
      \begin{align*}
        \pd{L}{r} &= mr\dot\theta^2 - kr
          & \pd{L}{\theta} &= 0 \\
        \pd{L}{\dot r} &= m\dot r
          & \pd{L}{\dot\theta} &= mr^2\dot\theta \\
        \frac{d}{dt}\pd{L}{\dot r} &= m\ddot r
          & \frac{d}{dt}\pd{L}{\dot \theta} &= 2mr\dot r\dot\theta + mr^2\ddot \theta
      \end{align*}

      Aplicando os resultados acima na equação de Euler-Lagrange associada à variável $r$, obtemos a equação de movimento associada a essa variável: $m\ddot r - mr\dot\theta^2 + kr = 0$.
      No caso da variável $\theta$, embora possamos proceder da mesma maneira (obtemos $2mr\dot r\dot\theta + mr^2\ddot \theta = 0$ nesse caso), o fato de que $\pd{L}{\theta} = 0$ implica que também $\frac{d}{dt}\pd{L}{\dot \theta} = 0$, o que significa que $\pd{L}{\dot \theta}$ é uma constante.
      Ou seja, $mr^2\dot\theta = C$ é uma \emph{constante do movimento} (nesse tipo de movimento, essa constante é o momento angular).
      Podemos utilizar essa informação para eliminar $\dot\theta$ da equação de $r$, o que simplifica a solução da equação do movimento.
    \end{solution}
\end{question}