\begin{question}
    Quantos graus de liberdade tem duas partículas puntuais livre para mover no espaço?
    \begin{enumerate}
      \item 2
      \item 3
      \item 4
      \item 5
      \item 6 \rightanswer
      \item 7
    \end{enumerate}

    \begin{solution}
      Cada partícula livre tem $N = 3$ graus de liberdade.
      Portanto, o \emph{sistema} composto por duas partículas livres tem $2N = 6$ graus de liberdade.
      Afinal, precisamos conhecer as três coordenadas da primeira partícula, $(x_1, y_1, z_1)$, e as três coordenadas da segunda partícula, $(x_2, y_2, z_2)$.
      São seis números; seis coordenadas; seis graus de liberdade.  
    \end{solution}
\end{question}