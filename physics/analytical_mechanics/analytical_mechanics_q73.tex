\begin{question}
    Quantos graus de liberdade tem uma partícula puntual confinada a mover-se num plano?
    \begin{enumerate}
      \item 0
      \item 1
      \item 2 \rightanswer
      \item 3
      \item 4
    \end{enumerate}

    \begin{solution}
      Compare com a situação da questão anterior: agora, se associarmos o plano mencionado no enunciado com o plano $xy$, imediatamente percebemos que $z = 0$, pois a partícula está confinada a esse plano.
      Por isso, só precisamos saber $x$ e $y$ para localizá-la sem dúvida nesse plano.
      Portanto, apenas \emph{duas} coordenadas são necessárias.
      Ou seja, dois graus de liberdade.
      Outra forma de chegar a essa conclusão é começar com o fato de que uma partícula \emph{livre} no espaço tridimensional tem três graus de liberdade ($N = 3$).
      Mas ela não é livre; ao invés disso, sua posição está condicionada a uma equação: a equação de vínculo.
      Chamemos a quantidade de equações de vínculo por $m$.
      Então, $m = 1$ e, deste modo, a quantidade de graus de liberdade dessa partícula não-livre é $N - m = 2$.
    \end{solution}
\end{question}