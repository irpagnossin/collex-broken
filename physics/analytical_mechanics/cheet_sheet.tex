% -----------------------------------------------------------------------------------------
% Encoding: UTF-8
% -----------------------------------------------------------------------------------------
% Autor:    Ivan Ramos Pagnossin
% Idioma:   pt-BR
% Licença:  Creative Commons BY-SA International 4.0
% -----------------------------------------------------------------------------------------
\documentclass[a4paper,10pt,oneside]{scrartcl}
	
	\usepackage[utf8]{inputenc}
\usepackage{lmodern}
\usepackage[T1]{fontenc}
\usepackage[brazil]{babel}
%\usepackage{tikz}\usetikzlibrary{calc,arrows}

\usepackage{icomma}
\usepackage{amsmath}
\usepackage{amssymb}
\usepackage{xspace}
%\usepackage{amsfonts}
\usepackage[locale=FR]{siunitx}
\usepackage{booktabs}
\usepackage{hyperref}
\usepackage[margin=2.5cm]{geometry}
\usepackage{graphicx}\graphicspath{{assets/}}
\usepackage{icomma}
\usepackage{lastpage}
\usepackage{comment}

\usepackage{irpagnossin-basic}
%\usepackage[answer,solution]{irpagnossin-exam}
\usepackage{irpagnossin-exam}

\usepackage{fancyhdr}
\fancyhead{}
\fancyfoot{}
\lfoot{}
\renewcommand{\headrulewidth}{0pt}
\pagestyle{fancy}

\rfoot{\footnotesize página \thepage\ de \pageref{LastPage}}

% Comandos comuns
\let\emph=\textbf
%\let\vec=\mathbf
%\let\hat=\mathbf
\newcommand\rightanswer{\textcolor{magenta}{$\leftarrow$(resposta correta)}}
\newcommand\vmeasure[2]{\ensuremath{#1}~\unit{#2}}

\newcommand\ava[1]{``\texttt{#1}''}
\newcommand\pd[2]{\ensuremath{\frac{\partial #1}{\partial #2}}}

\newcommand\half{\ensuremath{\frac{1}{2}}}

\NewEnviron{production}[1][]{%
	\color{blue!50!black}
  \vspace{\questionskip}
  \noindent
  \begingroup
    \bfseries\upshape\selectfont
    \color{blue!50!black}
    \MakeUppercase{orientações para produção}%
  \endgroup
  \begingroup
  	\footnotesize
    \ifthenelse{\equal{#1}{}}{}{~(#1)}%
  \endgroup
  \par
  \addpenalty{+300}%
  \footnotesize
  \noindent\BODY
  \par
}{%
  \addpenalty{-300}%
}

\NewEnviron{choices}[1][]{%
  \vspace{\questionskip}
  \color{blue!50!black}
  \noindent
  \begingroup
    \footnotesize
    \bfseries\upshape\selectfont
    \color{blue!50!black}
    \MakeUppercase{alternativas}%
  \endgroup
  \begingroup
    \footnotesize
    \ifthenelse{\equal{#1}{}}{}{~(#1)}%
  \endgroup
  \par
  \addpenalty{+300}%
  \footnotesize
  \noindent\BODY
  \par
}{%
  \addpenalty{-300}%
}


\newcommand\centeredfigure[2]{\begin{center}{\includegraphics[width=#1]{#2}}\end{center}}

\DeclareMathOperator{\sen}{sen}
\let\sin=\sen

	\title{Formulário de Física}
	\author{Prof. Dr. Ivan Ramos Pagnossin}
	\date{\today}

	\lhead{\footnotesize Formulário de Física}
	\rhead{\footnotesize Prof. Dr. Ivan Pagnossin, \today}

\begin{document}

	\section*{Rotações}
	\begin{gather*}
		\measure{180}{\degree} = \measure{\pi}{rad},
			\quad
		\omega = 2\pi/T \\
		s = \theta r, \quad v = \omega r, \quad a = \alpha r \\
		\text{$\alpha$ constante: }\omega(t) = \theta_0 + \alpha t, \quad \theta(t) = \theta_0 + \omega_0 t + \frac{1}{2} \alpha t^2 \quad\text{e}\quad \omega^2 = \omega_0^2 + 2\alpha \Delta\theta \\
		I = \sum_i m_i r_i^2 \quad\text{e}\quad dI = r^2\, dm\\
		\text{Teorema dos eixos paralelos: } I = I_\text{CM} + md^2 \\
		\text{Momento angular: }\vec L = I \vec\omega = m \vec r \times \vec v,
			\quad
		|\vec L| = mrv\sin\theta \\
		\text{Torque} = \vec\tau = \frac{d\vec L}{dt} = \vec F \times \vec r,
			\qquad
		|\vec \tau| = F r \sin\theta = I\alpha \\
		T_\text{rotação} = \frac{1}{2} I \omega^2, \quad T_\text{translação} = \frac{1}{2} m v^2
	\end{gather*}

	\section*{Forças inerciais}
	\begin{gather*}
		\vec F - m \vec A = m \vec a \\
		\vec F_\text{centrífuga} = -m \vec\omega \times \left(\vec\omega \times \vec{r'}\right), \qquad
			F_\text{centrífuga} = m r' \omega^2 \sin\theta \\
		\vec F_\text{Coriolis} = - 2 m \vec\omega \times \vec{v'}, \qquad F_\text{Coriolis} = 2 m v' \omega \sin\phi
	\end{gather*}

	\section*{Coordenadas curvilíneas}
	\begin{gather*}
		\text{Coordenadas retangulares: } ds^2 = dx^2 + dy^2 + dz^2 \\
		\text{Coordenadas polares: } ds^2 = dr^2 + r^2\, d\theta^2 \\
		\text{Coordenadas cilíndricas: } ds^2 = dr^2 + r^2\, d\theta^2 + dz^2 \\
		\text{Coordenadas esféricas: } ds^2 = d\rho^2 + \rho^2 \, d\phi^2 + \rho^2 \sin^2\theta\, d\theta^2 \\
		L = \int_0^\lambda d\tau\, \sqrt{\left|\sum_{i}\sum_{j} g_{ij} \frac{dq^i}{d\tau} \frac{dq^i}{d\tau}\right|} \\
		\vec b_i = \vec\nabla q^i \text{ (contra-variante) e } 
		\vec b_i^{\star} = \partial\vec r/\partial q^i \text{ (covariante)}\\
		g_{ij} = \vec b_i^{\star} \cdot \vec b_j^{\star}
			\quad\text{e}\quad
		\delta_{ij} = \vec b_i \cdot \vec b_j^{\star} \\
		v^i = \frac{dq^i}{dt} = \dot q^i,
			\quad
		v_i = \sum_j g_{ij} v^j
			\quad\text{e}\quad
		|\vec v| = \sqrt{\sum_{i} v^i v_i}
	\end{gather*}

	\section*{Formalismo lagrangeano}
	\begin{gather*}
		T = \frac{1}{2} mv^2 = \frac{1}{2} m \left(\dot x^2 + \dot y^2 + \dot z^2 \right),
			\quad
		U(\vec r) := \int_{\text{ref}}^{\vec r} \vec F \cdot d\vec\ell \\
		L = T - U \\
		\frac{d}{dt}\frac{\partial T}{\partial \dot q^i} - \frac{\partial T}{\partial q^i} = F_i 
			\quad\text{ou}\quad
		\frac{d}{dt}\frac{\partial L}{\partial \dot q^i} - \frac{\partial L}{\partial q^i} = 0
			\text{ com }
		F_i = \frac{\partial U}{\partial q^i}
	\end{gather*}

\end{document}

\frac{d}{dt}\frac{\partial T}{\partial \dot q^i} - \frac{\partial T}{\partial q^i} = \sum_{j = 1}^n \lambda_j \frac{\partial f_j}{\
		partial q^i} \\
		\frac{d}{dt}\frac{\partial T}{\partial \dot q^i} - \frac{\partial T}{\partial q^i} = F_i + \sum_{j = 1}^n \lambda_j \frac{\partial f_j}{\partial q^i}\\
		\delta W = F_x\, \delta x + F_y\, \delta y + F_z\, \delta z, \qquad \delta W = \sum_i F_i\, \delta q^i \\