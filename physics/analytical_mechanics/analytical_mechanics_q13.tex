\begin{question}
    Duas rodas denteadas, de raios \measure{10}{cm} e \measure{4}{cm}, estão conectadas por uma correia, de modo que ambas giram sem deslizar.
    Determine a velocidade angular da roda menor, sabendo que a roda maior gira a \measure{60}{rpm}.

    \begin{answer}
      \measure{150}{rpm} ou \measure{1.57e1}{rad/s}
    \end{answer}

    \begin{solution}
      Para que não haja deslizamento entre as rodas, a velocidade linear de ambas no ponto de contato deve ser a mesma.
      Ou seja, $v_1 = v_2 \Rightarrow \omega_1 R_1 = \omega_2 R_2$.
      Escolha a roda 1 como a menor, por exemplo.
      Então, $\omega_1 = \omega_2 R_2/R_1 = 60 \cdot 10 / 4 = \measure{150}{rpm}$ (ou \measure{1.57e1}{rad/s}).
    \end{solution}
\end{question}