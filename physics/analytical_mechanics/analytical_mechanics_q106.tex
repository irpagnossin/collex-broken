\begin{question}
  	Considere novamente o cenário do malabaris (questão anterior).
  	Determine a expressão da energia potencial gravitacional em termos das coordenadas $(X,Y,\theta)$.

  	\begin{enumerate}
  		\item $U = (m_1 + m_2)gY$ \rightanswer
  		\item $U = -(m_1 + m_2)gY$
  		\item $U = m_1gy_1 + m_2gy_2$
  		\item $U = - m_1gy_1 - m_2gy_2$
  		\item $U = (m_1 + m_2)gY + \half I \dot\theta^2$
  	\end{enumerate}

    \begin{solution}
      A energia potencial gravitacional pode ser facilmente expressa em termos da altura do centro de massa, $Y$, e da massa $M$ do sistema: $U = MgY = (m_1 + m_2)gY$.
    \end{solution}
\end{question}