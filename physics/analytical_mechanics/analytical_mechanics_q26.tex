\begin{question}
    Determine o momento de inércia de dois objetos puntuais, cada um com massa \measure{1.00}{kg} e que orbitam um eixo, a \measure{2.00}{m} dele.

    \begin{answer}
      \measure{8.00}{kg.m^2}
    \end{answer}

    \begin{solution}
      O momento de inércia de um sistema de partículas é dado por $I = \sum_i m_i r_i^2$, onde a somatória é feita nas partículas presentes. $m_i$ é a massa da $i$-ésima partícula e $r_i$, sua distância com relação ao eixo com relação ao qual se quer calcular $I$.
      Então, no caso particular deste problema, temos: $I = m_1 r_1^2 + m_2 r_2^2 = 2 mr^2$, pois $m_1 = m_2 = m = \measure{1.00}{kg}$ e $r_1 = r_2 = r = \measure{2.00}{m}$.
      Consequentemente, $I = \measure{8.00}{kg.m^2}$.
    \end{solution}
\end{question}