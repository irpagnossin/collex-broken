\begin{question}
    Considere uma partícula movendo-se num plano ao longo da trajetória dada por $q^1 \equiv r(t) = 3$ e $q^2 \equiv \theta(t) = 2t$, onde $(q^1, q^2) \equiv (r,\theta)$ é um sistema de coordenadas polar e $t$ representa o tempo.
    Determine as componentes covariantes do vetor de velocidade $v_i := \sum_{j} g_{ij} v^j$ (é comum omitir o símbolo de somatória quando um índice aparece repetido na expressão, uma vez sobrescrito e outra vez subscrito, como é o caso de $j$. Assim, podemos escrever $v_i := g_{ij} v^j$. Essa é a chamada ``convenção de Einstein'').
    \begin{enumerate}
      \item $v_1 = 0$ \rightanswer
      \item $v_2 = 18$ \rightanswer
      \item $v^1 = 0$
      \item $v^2 = 2$
      \item $v_1 = 18$
    \end{enumerate}

    \begin{solution}
      $q^1 = 3 \Rightarrow v^1 \equiv \dot q^1 = 0$ e $q^2 = 2t \Rightarrow v^2 \equiv \dot q^2 = 2$.
      Consequentemente,
      \begin{align*}
        v_1 = g_{11} v^1 + g_{12} v^2 &= 1 \cdot 0 + 0 \cdot 2 = 0 \\
        v_2 = g_{21} v^1 + g_{22} v^2 &= 0 \cdot 0 + r^2 \cdot 2  = 18.
      \end{align*}
    \end{solution}
\end{question}