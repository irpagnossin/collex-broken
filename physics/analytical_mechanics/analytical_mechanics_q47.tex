\begin{question}
    Uma pequena esfera homogênea, de raio \measure{2}{cm}, rola sem deslizar por uma mesa horizontal, com velocidade linear de \measure{50.0}{cm/s}, e sobe por uma rampa.
    Qual é a altura máxima que o centro de massa dessa esfera, medida a partir do nível da mesa, atinge antes de regressar?
    A aceleração da gravidade é de \measure{9.81}{m/s^2}.

    \begin{answer}
      \measure{3.78}{cm} (100\% do valor do item) ou \measure{1.78}{cm} (80\% do valor do item).
    \end{answer}
    
    \begin{solution}
      Considere a figura abaixo, que ilustra as duas situações: inicial ($A$) e final ($B$).

      \centeredfigure{0.7\textwidth}{20180821_214855}
      
      A energia mecânica é a soma das energias (i) cinética de translação do centro de massa, (ii) cinética de rotação em torno do centro de massa e (iii) potencial gravitacional.
      
      Em $A$ temos essas três formas de energia, então: $E_A = \half m v^2 + \half I \omega^2 + mgr$, onde $m$, $r$ e $I = \frac{2}{5} mr^2$ são a massa, o raio e o momento de inércia da esfera, respectivamente.
      $v$ é a velocidade de translação do centro de massa, $\omega$ é a velocidade angular de rotação e $g$ é a aceleração da gravidade.
      Note que, nessa expressão, escolhemos, conveninentemente, medir a energia potencial gravitacional a partir do nível da mesa.

      Na situação final $B$, a esfera está parada (não gira nem desloca-se), de modo que a energia está toda na forma potencial gravitacional: $E_B = mgh$, onde $h$ é a altura que caracteriza essa situação (veja a figura).

      Como não há deslizamento, $v = \omega r$.
      Além disso, por esse mesmo motivo a força de atrito, responsável por fazer girar a esfera, não realiza trabalho.
      Consequentemente, a energia é conservada (pois a força gravitacional é conservativa).
      Então, $E_A = E_B$.
      Usando as expressões anteriores para a energia mecânica em $A$ e $B$ e resolvendo para $h$, concluimos que $h = r + 7v^2/(10g)$.
      Ou seja, $h = \measure{3.78}{cm}$.
    \end{solution}
\end{question}