\begin{question}
  	Determine a equação de movimento do sistema ilustrado abaixo em termos da coordenada $\theta$, em que uma barra de comprimento $l$ e massa $m$, uniformemente distribuída, é apoiada no ponto $P$, no chão, e liberada para cair sob a ação da gravidade $g$.
  	Considere que o atrito em $P$ é suficiente para evitar que a barra escorregue (\ie, o ponto $P$ permanece imóvel).

		\begin{center}
      \includegraphics[width=0.4\textwidth]{20180719_121232}
    \end{center}

  	\begin{enumerate}
  		\item $2l \ddot \theta - 3g\cos\theta = 0$
  		\item $2l \ddot \theta + 3g\cos\theta = 0$ \rightanswer
  		\item $l \ddot\theta - 6 g \cos\theta = 0$
  		\item $l \ddot\theta + 6 g \cos\theta = 0$
  		\item $2l \ddot \theta + 3g\sin\theta = 0$
  	\end{enumerate}

    \begin{solution}
      A energia cinética pode ser escrita na forma puramente rotacional se considerarmos a rotação em torno do eixo $P$.
      Nesse caso, $T = \half I_P \dot\theta^2$, onde $I_P$ é o momento de inércia com relação a $P$.
      Usando o fato de que o momento de inércia de uma barra de massa $m$, homogeneamente distribuída, com relação ao seu centro de massa é $I = \frac{1}{12} m l^2$, e usando o teorema dos eixos paralelos, podemos afirmar que $I_P = I + m(l/2)^2 = \frac{1}{6}ml^2$.
      Então, $T = \frac{1}{6}ml^2 \dot\theta^2$.
      Quanto à energia potencial gravitacional, $U = mgz$, onde $z$ é a altura do centro de massa, medida a partir do piso.
      Como $z = \frac{l}{2}\sin\theta$, então $U = \half mgl \sin\theta$ e a lagrangeana fica assim:
      \begin{equation*}
        L := T - U = \frac{1}{6}ml^2 \dot\theta^2 - \half mgl \sin\theta.
      \end{equation*}

      Em seguida calculamos as derivadas relevantes de $L$:
      \begin{equation*}
        \pd{L}{\theta} = -\half mgl\cos\theta,
        \qquad
        \pd{L}{\dot\theta} = \frac{1}{3}ml^2\dot \theta
        \qquad\text{e}\qquad
        \frac{d}{dt}\pd{L}{\dot\theta} = \frac{1}{3}ml^2\ddot \theta.
      \end{equation*}

      Usando essas expressões na equação de Euler-Lagrange para $\theta$, obtemos a equação do movimento: $\frac{1}{3}l\ddot \theta + \half g\cos\theta = 0$.
    \end{solution}
\end{question}