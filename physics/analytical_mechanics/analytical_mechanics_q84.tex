\begin{question}
    Ainda sobre o cenário da conta presa a um arame parabólico, qual é a expressão da energia potencial gravitacional, escrita em termos do único grau de liberdade?
    Represente a aceleração da gravidade por $g$.
    \begin{enumerate}
      \item $U = mg(ax)^2$ \rightanswer
      \item $U = mgz$ \rightanswer
      \item $U = mgh$
      \item $U = -mg(ax)^2$
      \item $U = 0$
    \end{enumerate}

    \begin{solution}
      Novamente, pensando incialmente em termos das coordenadas cartesianas, sabemos que a energia potencial gravitacional pode ser assim expressa no sistema de coordenadas da questão: $U = mgz$.
      Mas $z = ax^2$, então $U = mg(ax)^2$.
      As duas expressões de $U$ são válidas.
    \end{solution}
\end{question}