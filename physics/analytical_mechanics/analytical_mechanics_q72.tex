\begin{question}
    Quantos graus de liberdade tem uma partícula puntual livre para mover-se no espaço?
    \begin{enumerate}
      \item 0
      \item 1
      \item 2
      \item 3 \rightanswer
      \item 4
    \end{enumerate}

    \begin{solution}
      A quantidade de graus de liberdade tem a ver com a quantidade de coordenadas que precisamos ter à mão para localizar, sem dúvida, uma partícula no espaço.
      Como nesse caso a partícula pode estar em qualquer lugar do espaço, é conveniente imaginar um sistema de coordenadas cartesianas para fazer referência à sua posição.
      Nesse caso, precisamos conhecer as coordenadas $x$, $y$ e $z$.
      Ou seja, precisamos de \emph{três} coordenadas.
      Por isso, três graus de liberdade.
    \end{solution}
\end{question}