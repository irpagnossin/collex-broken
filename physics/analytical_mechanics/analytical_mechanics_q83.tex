\begin{question}
    Ainda sobre o cenário da conta presa a um arame parabólico, qual é a expressão da energia cinética, escrita em termos do único grau de liberdade?
    Represente a massa da conta por $m$.
    \begin{enumerate}
      \item $T = \half m \dot x^2 (1 + 4a^2 x^2)$ \rightanswer
      \item $T = \half m (\dot x^2 + \dot y^2 + \dot z^2)$
      \item $T = \half m (\dot x^2 + \dot z^2)$
      \item $T = \half m \dot x^2 (1 + a^2)$
    \end{enumerate}

    \begin{solution}
      Geralmente é mais fácil começar pelas coordenadas cartesianas $(x,y,z)$.
      Nesse caso, a energia cinética da conta já é bem conhecida: $T = \half m (\dot x^2 + \dot y^2 + \dot z^2)$.
      Porém, vimos na questão anterior que $y = 0$ e $z = ax^2$, e daí decorre que $\dot y = 0$ e $\dot z = 2ax \dot x$.
      Usando essas expressões em $T$, obtemos:
      \begin{equation*}
        T = \half m \left[\dot x^2 + 0^2 + (2ax\dot x)^2\right] =  \half m \dot x^2 (1 + 4a^2 x^2).
      \end{equation*}

      Seria igualmente válido escrever $T$ em termos apenas de $z$ ao invés de $x$, mas nesse caso a expressão seria mais complexa.
    \end{solution}
\end{question}