\begin{question}
    Determine as equações de movimento do sistema ilustrado abaixo em termos das coordenadas $(q_1, q_2)$, em que uma barra de comprimento $l$ e massa $m$, uniformemente distribuída, é suspensa na extremidade de uma mola de constante elástica $k$ e comprimento natural $\ell_0$, livre para oscilar verticalmente apenas.
    Considere que $g$ é a acelerãção da gravidade.

    \begin{center}
      \includegraphics[width=0.4\textwidth]{20180719_110930}
    \end{center}

    \begin{enumerate}
    	\item $m \ddot q_1 - \half m l \sin q_2 \ddot q_2 - \half m l \cos q_2 \dot q_2^2 - mg + k (q_1 - \ell_0) = 0$ \rightanswer
    	\item $\frac{1}{3} l \ddot q_2 - \half \sin q_2 \ddot q_1 + \half g \sin q_2 = 0$ \rightanswer
    	\item $m \ddot q_1 - \half m l \sin q_2 \ddot q_2 - \half m l \cos q_2 \dot q_2^2 - mg + k q_1 = 0$
    	\item $m \ddot q_1 - mg + k (q_1 - \ell_0) = 0$
    	\item $l \ddot q_2 + 2g \sin q_2 = 0$
    \end{enumerate}

    \begin{solution}
      Comecemos pela energia cinética.
      Há dois termos: a energia cinética de translação do centro de massa da barra e a energia cinética de rotação da barra em torno do seu centro de massa.
      Inicialmente, escrevemos $T = \half m (\dot x^2 + \dot y^2) + \half I \dot q_2^2$, onde $(x,y)$ são as coordenadas cartesianas do centro de massa e $I$ é o momento de inércia com relação ao centro de massa.
      Analisando a figura, podemos afirmar que $x = \half l \sin q_2$ e $y = -q_1 - \half l \cos q_2$.
      Derivando essas expressões, obtemos $\dot x = \half l \cos q_2 \dot q_2$ e $\dot y = -\dot q_1 + \half l \sin q_2 \dot q_2$.
      Em seguida, usamos esses resultados na expressão de $T$ para escrever a energia cinética em termos apenas de $q_1$ e $q_2$.
      Obtemos $T = \half m \dot q_1^2 + \frac{1}{8} m l^2 \dot q_2^2 + \half I \dot q_2^2 - \half m l \sin q_2 \dot q_1 \dot q_2$.
      Essa expressão pode ser simplificada usando o fato de que o momento de inércia da barra é $I = \frac{1}{12}ml^2$.
      Com isso obtemos $T = \half m \dot q_1^2 + \frac{1}{6} m l^2 \dot q_2^2 - \half m l \sin q_2 \dot q_1 \dot q_2$.

      Vejamos agora a energia potencial.
      Temos duas contribuições: a gravidade e a força elástica.
      A energia potencial gravitacional, medida a partir do nível $O$, é $U_g = mgy$ (lembre-se de que $y$ é a posição vertical do centro de massa).
      Mas vimos que $y = -q_1 - \half l \cos q_2$, então $U_g = -mg(q_1 + \half l \cos q_2)$.
      A energia potencial elástica é mais fácil: $U_e = \half k (q_1 - \ell_0)^2$.
      Então, a energia potencial do sistema é $U = U_g + U_e = \half k (q_1 - \ell_0)^2-mg(q_1 + \half l \cos q_2)$.

      Deste modo, a lagrangeana é:
      \begin{equation*}
        L := T - U = \half m \dot q_1^2 + \frac{1}{6} m l^2 \dot q_2^2 - \half m l \sin q_2 \dot q_1 \dot q_2 - \half k (q_1 - \ell_0)^2 + mg(q_1 + \half l \cos q_2).
      \end{equation*}

      Em seguida determinamos as derivadas relevantes de $L$ com relação a $q_1$ e a $q_2$:
      \begin{align*}
        \pd{L}{q_1} &= mg - k(q_1 - \ell_0)
          & \pd{L}{q_2} &= -\half m l \cos q_2 \dot q_1 \dot q_2 - \half mgl \sin q_2 \\
        \pd{L}{\dot q_1} &= m\dot q_1 - \half m l \sin q_2 \dot q_2
          & \pd{L}{\dot q_2} &= \frac{1}{3} ml^2 \dot q_2 - \half m l \sin q_2 \dot q_1 \\
        \frac{d}{dt}\pd{L}{\dot q_1} &= m\ddot q_1 - \half m l \sin q_2 \ddot q_2 - \half m l \cos q_2 \dot q_2^2
          & \frac{d}{dt}\pd{L}{\dot q_2} &= \frac{1}{3} m l^2 \ddot q_2 - \half m l \sin q_2 \ddot q_1 - \half m l \cos q_2 \dot q_1 \dot q_2
      \end{align*}

      Usando essas expressões nas equações de Euler-Lagrange, uma para $q_1$ e uma para $q_2$, obtemos as equações do movimento:
      \begin{equation*}
      m \ddot q_1 - \half m l \sin q_2 \ddot q_2 - \half m l \cos q_2 \dot q_2^2 - mg + k (q_1 - \ell_0) = 0
      \qquad\text{e}\qquad
      \frac{1}{3} l \ddot q_2 - \half \sin q_2 \ddot q_1 + \half g \sin q_2 = 0
      \end{equation*}
    \end{solution}
\end{question}