\begin{question}
    O elemento de comprimento $ds$ pode ser expresso, em coordenadas polares $(r, \theta)$, assim: $ds^2 = dr^2 + r^2 d\theta^2$.
    Selecione a(s) alternativa(s) que representa(m) componente(s) do tensor de métrica (não necessariamente todas).
    Para isso, note que a expressão acima para $ds^2$ é uma forma compacta de:
    \begin{equation*}
      (ds)^2 = 
      \begin{pmatrix}
        dr & d\theta
      \end{pmatrix}
      \begin{pmatrix}
        g_{rr} & g_{r\theta } \\
        g_{\theta r} & g_{\theta \theta }
      \end{pmatrix}
      \begin{pmatrix}
        dr \\
        d\theta
      \end{pmatrix} =
      \begin{pmatrix}
        dr & d\theta
      \end{pmatrix}
      \begin{pmatrix}
        1 & 0 \\
        0 & r^2
      \end{pmatrix}
      \begin{pmatrix}
        dr \\
        d\theta
      \end{pmatrix}
    \end{equation*}

    \begin{enumerate}
      \item $g_{rr} = 1$ \rightanswer
      \item $g_{r \theta} = 0$ \rightanswer
      \item $g_{\theta\theta} = r^2$ \rightanswer
      \item $g_{\theta r} = r^2$
      \item $g_{\theta\theta} = r^2d\theta^2$
      \item $g_{rr} = dr^2$
    \end{enumerate}

    \begin{solution}
      Basta comparar as duas expressões matriciais acima para concluir que $g_{rr} = 1$, $g_{r \theta} = g_{\theta r} = 0$ e $g_{\theta\theta} = r^2$.
      Equivalentemente, podemos escrever $g_{11} = 1$, $g_{12} = g_{21} = 0$ e $g_{22} = r^2$.
    \end{solution}
\end{question}