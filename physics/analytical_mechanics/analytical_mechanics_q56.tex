\begin{question}
    Uma circunferência de raio $R = 2$ pode ser representada, em coordenadas polares, pela curva $(q^1, q^2) \equiv (r, \theta) = (2, \tau)$, com $0 \le \tau \le 2\pi$.
    Determine o comprimento dessa curva.
    Para isso, proceda assim: 
    \begin{compactenum}
      \item Determine a derivada de cada componente da curva com relação ao parâmetro $\tau$.
      \item Identifique as componentes do tensor de métrica associado às coordenadas polares (veja os exercícios anteriores).
      \item Monte a expressão $L = \int_0^{2\pi} d\tau\, \sqrt{\left|\sum_{i=1}^3\sum_{j=1}^3 g_{ij} \frac{dx^i}{d\tau} \frac{dx^i}{d\tau}\right|}$, onde $L$ é o comprimento procurado.
      Note que $g_{ij} = 0$ quando $i \ne j$, o que simplifica bastante essa expressão.
      \item Resolva a expressão anterior para determinar $L$.
      \item Compare o resultado com aquele que você esperaria se resolvesse o problema utilizando seus conhecimentos de geometria.
    \end{compactenum}

    \begin{answer}
      12,6
    \end{answer}

    \begin{solution}
      \begin{compactenum}
        \item $\frac{dq^1}{d\tau} = \frac{d}{d\tau}(2) = 0$ e $\frac{dq^2}{d\tau} = \frac{d\tau}{d\tau} = 1$.
        \item $g_{11} = 1$, $g_{12} = g_{21} = 0$ e $g_{22} = (q^1)^2$.
        \item A soma $\sum_{i=1}^3\sum_{j=1}^3 g_{ij} \frac{dx^i}{d\tau} \frac{dx^i}{d\tau}$ pode se simplificada, antes de a escrevermos, observando que $g_{12} = g_{21} = 0$, de sorte que a somatória fica $g_{11}\left(\frac{dx^1}{d\tau}\right)^2 + g_{22}\left(\frac{dx^2}{d\tau}\right)^2 = 2 + (q^1)^2$.
        Mas como o cálculo é todo feito sobre a curva, sabemos que $q^1 = 2$.
        Então, a somatória fica simplesmente $2 + 2 = 4$.
        Assim, podemos escrever $L = \int_{0}^{2\pi} 2\, d\tau$.
        \item A integral é imediata: $L = \int_{0}^{2\pi} 2\, d\tau = 4\pi$
        \item O resultado acima coincide com aquele que obtemos da geometria: $2\pi r$, onde $r = 2$ é o raio.
       \end{compactenum}
    \end{solution}
\end{question}