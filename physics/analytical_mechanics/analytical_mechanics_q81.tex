\begin{question}
    Considere uma conta livre para mover-se ao longo de um arame com formato parabólico dado por $z = ax^2$, onde $a$ é uma constante e $(x,y,z)$ é um sistema cartesiano de coordenadas com $x$ e $y$ horizontais e $z$ apontando para cima.
    Quantos são os graus de liberdade da conta?

    \begin{center}
      \includegraphics[width=0.4\textwidth]{20180719_181453}
    \end{center}

    \begin{enumerate}
      \item 0
      \item 1 \rightanswer
      \item 2
      \item 3
      \item 4
    \end{enumerate}

    \begin{solution}
      A posição da conta sobre o arame pode ser univocamente especificada pela sua distância até o vértice da parábola (ponto mais baixo do arame).
      Portanto, há apenas um grau de liberdade.
      Método alternativo: se a conta fosse livre, ela teria $N = 3$ graus de liberdade.
      Mas ela está sujeita a $m = 2$ equações de vínculo: $y = 0$ e $z = ax^2$.
      Portanto, ela tem $N - m = 1$ grau de liberdade.
    \end{solution}
\end{question}