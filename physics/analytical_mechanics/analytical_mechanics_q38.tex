\begin{question}
    Considere um sistema de referência inercial e o sistema de coordenadas retangulares, munido da base ortonormal tradicional $(\hat i, \hat j, \hat k)$.
    Nesse sistema, num instante qualquer, uma partícula puntual localiza-se na posição \vmeasure{\hat i + 2\hat j}{m} e está sujeita à força \vmeasure{3\hat j}{N}, move-se com velocidade \vmeasure{-4\hat i + \hat j}{m/s}.
    Determine o \emph{vetor} torque, com relação à origem do sistema de referência, sobre essa partícula.

    \begin{answer}
      \vmeasure{3\hat k}{N.m}. No AVA, \ava{(0,0,3) N.m}
    \end{answer}
    
    \begin{solution}
      $\vec\tau = \vec r \times \vec F = (\hat i + 2\hat j) \times 3\hat j = 3 \hat k$.
      A velocidade é irrelevante.
    \end{solution}
\end{question}