\begin{question}
    Considere uma partícula movendo-se num plano ao longo da trajetória dada por $q^1 \equiv r(t) = 3t$ e $q^2 \equiv \theta(t) = \pi/2$, onde $(q^1, q^2) \equiv (r,\theta)$ é um sistema de coordenadas polar e $t$ representa o tempo.
    Determine o módulo do vetor de velocidade, $|\vec v| := \sqrt{\sum_{i} v^i v_i} = \sqrt{v^1 v_1 + v^2 v_2}$ (não é necessário fornecer a unidade).

    \begin{answer}
      3,00
    \end{answer}

    \begin{solution}
      \begin{equation*}
        \sqrt{v^1 v_1 + v^2 v_2} = \sqrt{3 \cdot 3 + 0 \cdot 0} = 3
      \end{equation*}
    \end{solution}
\end{question}