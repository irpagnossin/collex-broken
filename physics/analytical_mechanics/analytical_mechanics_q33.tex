\begin{question}
    Considere um sistema de referência inercial e o sistema de coordenadas retangulares, munido da base ortonormal tradicional $(\hat i, \hat j, \hat k)$.
    Nesse sistema, num instante qualquer, uma partícula puntual de \measure{1}{kg} localiza-se na posição \vmeasure{\hat i + 2\hat j}{m} e move-se com velocidade \vmeasure{-4\hat i + \hat j}{m/s}.
    Determine o \emph{vetor} momento angular dessa partícula com relação à origem do sistema de referência.

    \begin{answer}
      \vmeasure{9\hat k}{J.s}. No AVA, \ava{(0,0,9) J.s}
    \end{answer}
    
    \begin{solution}
      $\vec L = \vec r \times \vec p = (\hat i + 2\hat j) \times (-4\hat i + \hat j) = -4 \hat i \times \hat i - 8 \hat j \times \hat i + \hat i \times \hat j + 2\hat j \times \hat j = 8 \hat k + \hat k = 9 \hat k$, em que utilizamos as identidades $\hat i \times \hat j = \hat k$ e $\hat i \times \hat i = \hat j \times \hat j = \vec 0$.
    \end{solution}
\end{question}