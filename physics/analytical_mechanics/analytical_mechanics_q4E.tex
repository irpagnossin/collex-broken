\begin{question}
	No último domingo, papai Fulano levou seus filhos, Sicrano e Beltrano, para passear no carrossel do parque.
	Enquanto curtia o passeio ao lado de Beltrano (cada volta durava \measure{30.0}{s}), que estava num dos carros de bombeiro, a \measure{10.00}{m} de distância do centro, ele notou que Sicrano estava escorregando do cavalo, que localizava-se na mesma radial, mas a \measure{7.00}{m} do centro (veja a figura).
	Ao mover-se para ajudá-lo, a \measure{1.20}{m/s} (medido com relação à plataforma girante do carrossel), Fulano sentiu uma força estranha que o desviava de seu objetivo.
	Determine a intensidade e o sentido dessa força, sabendo que Fulano tem massa de \measure{85.0}{kg}.

	\begin{answer}
		\measure{42.7}{N} tangencial, no sentido do movimento do carrossel.
	\end{answer}

	\centeredfigure{0.5\textwidth}{20180804_213549}

	\begin{solution}
		Como a ``força estranha'' aparece quando Fulano \emph{se move}, trata-se da força de Coriolis (a outra candidata, a força centrífuga, depende da posição de Fulano no carrossel, não da velocidade), cuja expressão é $\vec F = 2m\vec v' \times \vec \omega$, onde $m$ é a massa do Fulano, $\vec v'$ é sua velocidade, medida no referencial que gira juntamente com o carrossel, e $\vec\omega$ é a velocidade angular do carrossel com relação a um sistema inercial (o chão, por exemplo).

		Analisando a situação, vemos que o vetor velocidade angular aponta para cima ($\vec\omega = \omega\hat k$: regra da mão direita para a representação vetorial de rotação). $\omega$ pode ser obtida da informação sobre o período $T = \measure{30}{s}$ da rotação: $\omega = 2\pi/T = 2\pi/30 = \measure{0.209}{rad/s}$. Por outro lado, $\vec v' = \measure{1.20}{m/s}$ é radial, sentido centro ($\vec v' = -v'\hat r$).
		Desse modo, concluímos que a força de Coriolis é tangencial, no sentido do movimento do carrossel: $\vec F = 2m(-v'\hat r)\times(\omega \hat k) = 2mv'\omega \hat \phi$. E desse modo calculamos a intensidade da força de Coriolis: $F = 2mv'\omega = \measure{42.7}{N}$.
	\end{solution}
\end{question}