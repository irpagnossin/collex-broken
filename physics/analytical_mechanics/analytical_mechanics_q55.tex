\begin{question}
    O elemento de comprimento $ds$ pode ser expresso, em coordenadas afins $(q^1, q^2, q^3)$, assim: $ds^2 = (dq^1)^2 + (dq^2)^2 + (dq^3)^2 + 2\cos\alpha dq^1 dq^2 + 2 \cos\beta\, dq^1 dq^3 + 2 \cos\gamma\, dq^2 dq^3$.
    Selecione a(s) alternativa(s) que representa(m) corretamente as componentes do tensor de métrica:
    \begin{enumerate}
      \item $g_{11} = 1$ \rightanswer
      \item $g_{22} = 1$ \rightanswer
      \item $g_{13} = \cos\beta$ \rightanswer
      \item $g_{33} = 0$
      \item $g_{22} = (dq^2)^2$
      \item $g_{21} = 0$
    \end{enumerate}

    \begin{solution}
      Procedendo como nas questões anteriores, vemos que $g_{11} = g_{22} = g_{33} = 1$.
      A novidade aqui é que há termos cruzados $dq^1 dq^2$, $dq^1 dq^3$ e $dq^2 dq^3$.
      O fator que multiplica $dq^1 dq^2$ é $2\cos\alpha$, então o termo $g_{12} = \cos\alpha$.
      Perceba que $g_{22}$ é igual à \emph{metade} do fator que multiplica $dq^1 dq^2$.
      Isso acontece porque, na verdade, $2\cos\alpha\, dq^1 dq^2$ advém da soma $\cos\alpha\, dq^1 dq^2 + \cos\alpha\, dq^2 dq^1$, que surge no produto matricial (veja a primeira questão).
      Além disso, $g_{21} = g_{12}$ (isso é verdade para todos os elementos de $g$: $g_{ij} = g_{ji}$).
      Analogamente, o fator que multiplica $dq^1 dq^3$ é $2 \cos\beta$, então $g_{13} = g_{31} = \cos\beta$.
      Finalmente, o fator que multiplica $dq^2 dq^3$ é $2 \cos\gamma$, então $g_{23} = g_{32} = \cos\gamma$.
      Resumindo,
      \begin{equation*}
        g = \begin{pmatrix}
          1          & \cos\alpha & \cos\beta  \\
          \cos\alpha & 1          & \cos\gamma \\
          \cos\beta  & \cos\gamma & 1
        \end{pmatrix}.
      \end{equation*}
    \end{solution}
\end{question}