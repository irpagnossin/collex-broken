\begin{question}
    Considere um sistema de referência inercial e o sistema de coordenadas retangulares, munido da base ortonormal tradicional $(\hat i, \hat j, \hat k)$.
    Nesse sistema, uma partícula puntual de \measure{1}{kg} move-se sobre a reta $y = \measure{3}{m}$ com velocidade \vmeasure{-4\hat i}{m/s}.
    \begin{enumerate}
      \item Determine a intensidade do momento angular dessa partícula com relação à origem do sistema de referência.
      \item Determine a intensidade do momento angular dessa partícula com relação à posição \vmeasure{3\hat j}{m}.
    \end{enumerate}

    \begin{answer}
      \begin{enumerate}
        \item \measure{12.0}{kg.m^2/s}
        \item \measure{0.00}{kg.m^2/s}
      \end{enumerate}
    \end{answer}
    
    \begin{solution}
      \begin{enumerate}
        \item $L \equiv |\vec L| = |\vec r \times \vec p| = r m v \sin\theta = m v y = \measure{12}{kg.m^2/s}$, em que usamos o fato de que $y = r \sin\theta$.
        Vetorialmente, $\vec L = 12\hat k$, em unidades do SI.
        \item Nesse caso, $\vec r \parallel \vec p$.
        Consequentemente, $L = 0$.
        Note, portanto, que não podemos falar em momento angular sem fazer referência ao ponto com relação ao qual ele é calculado.
        Quando essa informação é omitida, geralmente o ponto em questão é o centro de massa.
      \end{enumerate}
    \end{solution}
\end{question}