\begin{question}
    Considere as coordenadas polares $(r,\theta)$, que relacionam-se com as coordenadas cartesianas $(x,y)$ por meio das equações:
    \begin{equation}
      \left\{
      \begin{matrix}
        x \equiv x(r, \theta) = r \cos\theta \\
        y \equiv y(r, \theta) = r \sin\theta
      \end{matrix}
      \right.
      \qquad\text{e}\qquad
      \left\{
      \begin{matrix}
        r \equiv r(x,y) = \sqrt{x^2 + y^2} \\
        \theta \equiv \theta(x,y) = \arctan\left(y/x\right)
      \end{matrix}
      \right.
    \end{equation}

    Utilize as expressões acima para determinar os produtos escalares $\vec b_i \cdot \vec b_j^{\star} = \delta_{ij}$.
    Selecione as alternativas corretas:
    \begin{enumerate}
      \item $\vec b_1 \cdot \vec b_1^{\star} = 1$ \rightanswer
      \item $\vec b_2 \cdot \vec b_2^{\star} = 1$ \rightanswer
      \item $\vec b_2 \cdot \vec b_1^{\star} = 1$
      \item $\vec b_2 \cdot \vec b_2^{\star} = 0$
      \item $\vec b_2 \cdot \vec b_3^{\star} = 1$
    \end{enumerate}

    \begin{solution}
      Pela definição, $\vec b_1 \cdot \vec b_1^{\star} = \delta_{11} = 1$, $\vec b_2 \cdot \vec b_2^{\star} = \delta_{22} = 1$ e $\vec b_1 \cdot \vec b_2^{\star} = \delta_{12} = 0$.
      Você também pode usar os $\vec b_i$ e $\vec b_i^\star$, determinados nas questões anteriores, e checar explicitamente essas identidades.
    \end{solution}
\end{question}