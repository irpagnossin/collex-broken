\begin{question}
  	Considere novamente o cenário do malabaris.
  	Além das forças de vínculo, que devem ser ignoradas, a única força presente é a gravitacional, que é conservativa.
  	Então, podemos utilizar a ``forma 1'' da equação de Lagrange,
  	\begin{equation*}
  		\frac{d}{dt}\pd{L}{\dot q^i} - \pd{L}{q^i} = 0,
  	\end{equation*}
  	onde $L = T - U$ é a função lagrangeana, para determinar as \emph{três} equações do movimento.
  	Faça isso.

  	\begin{enumerate}
  		\item $\ddot X = 0$ \rightanswer
  		\item $\ddot Y = -g$ \rightanswer
  		\item $\ddot\theta = 0$ \rightanswer
  		\item $\ddot Y = 0$
  		\item $\dot\theta = \text{constante}$
  		\item $\dot X = \text{constante}$
  	\end{enumerate}

    \begin{solution}
      Utilizando $T = \half M (\dot X^2 + \dot Y^2) + \half I \dot\theta^2$ e $U = MgY$, com $M = m_1 + m_2$, que obtivemos nas duas questões anteriores, escrevemos a lagrangeana: $T := T - U = \half M (\dot X^2 + \dot Y^2) + \half I \dot\theta^2 - MgY$.
      Em seguida determinamos as derivadas relevantes de $L$:
      \begin{align*}
        \pd{L}{X} &= 0
          & \pd{L}{Y} &= -Mg
            & \pd{L}{\theta} &= 0 \\
        \pd{L}{\dot X} &= M\dot X
          & \pd{L}{\dot Y} &= M\dot Y
            & \pd{L}{\dot \theta} &= I\dot \theta \\
        \frac{d}{dt}\pd{L}{\dot X} &= M\ddot X
          & \frac{d}{dt}\pd{L}{\dot Y} &= M\ddot Y
            & \frac{d}{dt}\pd{L}{\dot \theta} &= I\ddot \theta.
      \end{align*}

      Usando essas expressões nas equações de Euler-Lagrange, uma para $X$, uma para $Y$ e uma para $\theta$, obtemos as equações do movimento associadas a cada coordenada generalizada: $\ddot X = 0$, $\ddot Y = -g$ e $\ddot \theta = 0$.
      Note a presença da constante do movimento $M\dot X$ (momento linear), devido a $\pd{L}{X} = 0$.
      Idem para $I\dot\theta$ (momento angular), devido a $\pd{L}{\theta} = 0$.
    \end{solution}
\end{question}