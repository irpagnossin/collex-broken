\begin{question}
    Suponha que aproximemos a Terra por uma esfera de raio \measure{6.40e3}{km} e considere que ela dá uma volta em torno do seu próprio eixo em 23 horas, 56 minutos e 4 segundos (dia sideral).
    Nesse caso, qual é a velocidade linear, com relação ao seu centro de massa, de um ponto qualquer sobre o equador?

    \begin{answer}
      \measure{4.67e2}{m/s} ou \measure{1.68e3}{km/h}
    \end{answer}

    \begin{solution}
      A velocidade angular $\omega$ da Terra relaciona-se com a velocidade linear $v$ na superfície através da equação $v = R\omega$, onde $R$ é o raio da Terra.
      Como $\omega = 2\pi/86164 = \measure{7.29e-5}{rad/s}$, segue daí que $v = \measure{4.67e2}{m/s}$ (ou \measure{1.68e3}{km/h}).
    \end{solution}
\end{question}