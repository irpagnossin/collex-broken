\begin{question}
    Quantos graus de liberdade tem uma partícula puntual confinada a mover-se numa curva?
    \begin{enumerate}
      \item 0
      \item 1 \rightanswer
      \item 2
      \item 3
      \item 4
    \end{enumerate}

    \begin{solution}
      Por simplicidade, imagine que a curva mencionada é uma reta.
      Nesse caso, podemos escolher um sistema cartesiano em que essa reta resida no plano $xy$.
      Assim, podemos afirmar que $z = 0$ e que $x$ e $y$ relacionam-se por uma equação dessa forma: $ax + by = c$, onde $a$, $b$ e $c$ são constantes (essa é a equação implícita de uma reta).
      Logo, temos $m = 2$ equações de vínculo.
      Então, a quantidade de graus de liberdada da partícula confinada na reta é $N - m = 1$ ($N = 3$ é a quantidade de graus de liberdade de uma partícula livre no espaço tridimensional).
      Outra forma de avaliar esse cenário, que agora pode ser facilmente estendido para qualquer curva, é o seguinte: escolha um ponto sobre a curva e chame-o de referência, isto é, meça as distâncias sobre a curva a partir desse ponto.
      Procedendo assim, podemos localizar univocamente um ponto dessa curva pela distância orientada sobre a curva.
      Ou seja, só precisamos de um número, ou coordenada, para localizar o ponto (e a partícula sobre ele).
      Logo, temos apenas um grau de liberdade.
    \end{solution}
\end{question}