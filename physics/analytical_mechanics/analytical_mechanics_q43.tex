\begin{question}
    Determine o torque necessário para acelerar uniformemente uma roda denteada de alumínio do repouso até \measure{2000}{rpm} em \measure{10}{s}.
    Considere que essa roda denteada pode ser aproximada por um cilindro de raio \measure{10}{cm} e altura \measure{1}{cm}.
    A densidade do alumínio é de \measure{2.70e3}{kg/m^3} (cf. \href{https://www.webelements.com/aluminium/}{webelements.com}).

    \begin{answer}
      \measure{8.88e-2}{N.m}
    \end{answer}
    
    \begin{solution}
      O torque $\tau$ relaciona-se com a aceleração angular $\alpha$ desta forma: $\tau = I\alpha$, onde $I = \half mr^2$ é o momento de inércia de um disco de raio $r$ e massa $m$.
      A massa do cilindro é dada por $m = \pi r^2 h \rho$, onde $h$ é a altura e $\rho$, sua densidade.
      Consequentemente, $I = \frac{\pi}{2} r^4 h \rho = \measure{4.24e-3}{kg.m^2}$.
      A aceleração angular é simplesmente $\alpha = \measure{2000}{rpm}/\measure{10}{s} = \measure{20.9}{rad/s^2}$.
      Então, $\tau = \measure{8.88e-2}{N.m}$.
    \end{solution}
\end{question}