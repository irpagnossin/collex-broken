\begin{question}
    Considere que o momento de inércia do rotor principal de um helicóptero é de \measure{1.30e3}{kg.m^2}.
    Determine a intensidade do torque que o motor deve aplicar nesse rotor de modo a acelerá-lo uniformemente desde o repouso até sua velocidade de operação, \measure{300}{rpm}, em \measure{1.00}{min}.
    \begin{answer}
      \measure{6.80e2}{N.m}
    \end{answer}
    
    \begin{solution}
      Sabemos que $\tau = I\alpha$, onde $\tau$ é o torque necessário para causar uma aceleração angular $\alpha$ num objeto cujo momento de inércia é $I$.
      No caso apresentado, $\alpha = \measure{300}{rpm}/\measure{1.00}{min} = \measure{5.23e-1}{rad/s^2}$.
      E como $I = \measure{1.30e3}{kg.m^2}$, decorre imediatamente que $\tau = \measure{6.80e2}{N.m}$.
    \end{solution}
\end{question}