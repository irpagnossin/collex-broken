\begin{question}
    Uma força é aplicada tangencialmente à borda de uma polia que tem \measure{10}{cm} de raio e momento de inércia de \measure{1.00e-3}{kg.m^2} em relação ao seu eixo.
    A força tem módulo variável com o tempo, segundo a relação $F(t) = 0,5t + 0,3t^2$, com $F$ em newtons e $t$ em segundos.
    A polia está inicialmente em repouso.
    Em $t = \measure{3}{s}$, qual é sua aceleração angular?

    \begin{answer}
      \measure{420}{rad/s^2}
    \end{answer}
    
    \begin{solution}
      Como a força é aplicada perpendicularmente ao raio da polia, $\tau = rF$.
      E como $\tau = I\alpha$, então $\alpha(t) = rF(t)/I$.
      Em $t = \measure{3}{s}$, $F = \measure{4.2}{N}$.
      Assim, $\alpha = \measure{420}{rad/s^2}$ nesse instante.
    \end{solution}
\end{question}