\begin{question}
	Considere o conjunto de cinco objetos puntiformes ilustrados abaixo, unidos por hastes rígidas de massa desprezível e comprimento $d = \measure{0.350}{m}$.
	Cada objeto periférico tem massa igual a \measure{1.25}{kg}, enquanto o objeto central tem massa \measure{2.500}{kg}.
	Determine o momento de inércia desse sistema com relação ao eixo que passa pelo ponto $M$, perpendicularmente ao plano da figura.

	\centeredfigure{0.4\textwidth}{20180804_185334}

	\begin{answer}
		\measure{1.07}{kg.m^2}
	\end{answer}

	\begin{solution}
		O momento de inércia é dado por $I = \sum_i m_i r_i^2$, onde $m_i$ é uma massa puntual situada à distância $r_i$ do eixo com relação ao qual se quer calcular $I$.
		É bem fácil determinar o momento de inércia com relação ao centro de massa: $I_{CM} = 4md^2$, onde $m$ é a massa de cada objeto, de 1 a 4 (o objeto 5, no centro, não aparece nessa expressão porque sua distância até o centro é zero: $r_5 = 0$).
		Por isso, é mais fácil determinar o momento de inércia com relação ao eixo solicitado por meio do teorema dos eixos paralelos: $I_M = I_{CM} + MD^2$, onde $M = 6m$ é a massa do sistema todo (pois o objeto no centro tem o dobro da massa dos demais) e $D = \sqrt{2}d/2$ é a distância do centro até o ponto $M$.
		Então,
		\begin{equation*}
			I_M = 4md^2 + (6m)\left(\sqrt{2}d/2\right)^2 = 7md^2 = \measure{1.07}{kg.m^2}.
		\end{equation*}
	\end{solution}
\end{question}

