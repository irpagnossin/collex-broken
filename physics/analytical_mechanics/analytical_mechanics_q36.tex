\begin{question}
    O momento de inércia $I$ de um certo objeto, numa base ortonormal $(\hat i, \hat j, \hat k)$, é dado por
    \begin{equation}
      I =
        \begin{pmatrix}
          1 & 0 & 0 \\
          0 & 1 & 0 \\
          0 & 0 & 2
        \end{pmatrix},
    \end{equation}
    em \unit{kg.m^2}.
    Determine o momento angular $\vec L$ desse objeto, sabendo que sua velocidade angular é de $\vec \omega = \vmeasure{(\hat i + \hat j + \hat k)}{rad/s}$.

    \paragraph{Observação:} note que $\vec L$ não é paralelo a $\vec \omega$ nesse caso.
    Isso acontece porque $\vec\omega$ não é paralelo a qualquer um dos eixos principais ($\hat i$, $\hat j$ ou $\hat k$).

    \begin{answer}
      \vmeasure{(\hat i + \hat j + 2\hat k)}{J.s}. No AVA, \ava{(1,1,2)}, sem as aspas.
    \end{answer}
    
    \begin{solution}
      Primeiramente, escrevemos $\vec\omega$ na forma matricial: $\vec\omega = \begin{pmatrix}1\\1\\1\end{pmatrix}$.
      Em seguida,
      \begin{equation*}
        \vec L = I \cdot \vec\omega =
          \begin{pmatrix}
            1 & 0 & 0 \\
            0 & 1 & 0 \\
            0 & 0 & 2
          \end{pmatrix}
          \begin{pmatrix}
            1 \\ 1 \\ 1
          \end{pmatrix}
        = \begin{pmatrix}
            1 \\ 1 \\ 2
          \end{pmatrix}.
      \end{equation*}
      Finalmente, interpretamos $\vec L = \begin{pmatrix}1\\1\\2\end{pmatrix}$ como o vetor $\vec L = \hat i + \hat j + 2\hat k$.
    \end{solution}
\end{question}