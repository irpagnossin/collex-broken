\begin{question}
	Suponha que o movimento de uma partícula seja dado pela equação horária $x(t) = 4t^2 - 3$.
	\begin{inlineenum}
		\inlineitem Determine a expressão da velocidade e da aceleração dessa partícula em qualquer instante de tempo $t$.
		\inlineitem Se $x$ é dado em metros (\unit{m}) e $t$, em segundos (\unit{s}), quais devem ser as unidades das constantes $4$ e $3$ presentes na expressão de $x(t)$?
		\inlineitem Qual é a posição, velocidade e aceleração dessa partícula em $t = \measure{3}{s}$?
	\end{inlineenum}

	\begin{answer}
		\begin{enumerate}
			\item $v(t) = 8t$ e $a(t) = 8$.
			\item \measure{4}{m/s^2} e \measure{3}{m}.
			\item Em $t = \measure{3}{s}$, $x(3) = \measure{33}{m}$, $v(3) = \measure{24}{m}$ e $a(3) = \measure{8}{m/s^2}$.
		\end{enumerate}
	\end{answer}

	\begin{solution}
		\begin{enumerate}
			\item A equação horária da velocidade da partícula é dada pela derivada de $x(t)$ com relação ao tempo: $v(t) := \frac{dx}{dt} = 8t$.
			Similarmente, a aceleração é a derivada de $v(t)$: $a(t) := \frac{dv}{dt} = 8$.
			\item O lado esquerdo da equação horária $x(t)$ é dado em metros (no Sistema Internacional de Unidades).
			Por isso, o lado direito também tem de ser dado em metros e, particularmente, cada parcela, separadamente, deve ser dada em metros.
			Então, a constante $3$ deve ser dada em metros: \measure{3}{m}.
			Do mesmo modo, $[4t^2] = \unit{m}$.
			Mas $[t^2] = \unit{s^2}$, o que significa que $[4] = \unit{m/s^2}$.
			Ou seja, \measure{4}{m/s^2}.
			\item A posição em $t = \measure{3}{s}$ é dada por $x(3) = \measure{33}{m}$.
			Analogamente, $v(3) = \measure{24}{m/s}$ é a velocidade e $a(3) = \measure{8}{m/s^2}$ é a aceleração.
		\end{enumerate}
	\end{solution}
\end{question}