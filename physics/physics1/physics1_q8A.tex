\begin{question}
	Lembre-se da sua última viagem pelo terreno acidentado do sul de Minas Gerais.
	Ao passar de carro pelo topo de um pequeno morro, que tem o perfil de uma circunferência de raio $R = \measure{200}{m}$ (figura), você nota uma sensação de perda parcial de peso.
	Isso acontece porque para que o carro acompanhe o perfil da pista (um movimento circular naquele ponto, note), deve haver uma resultante que atue como força centrípeta.
	Como consequência, a força normal, que é a responsável pela nossa sensação de peso, é menor do que seria numa pista plana.
	Sabendo disso, com que volocidade você deve passar no próximo morro (suponha que ele tenha o mesmo raio) para que você tenha a sensação de ausência de peso?
	Considere que a aceleração da gravidade é de \measure{9.81}{m/s^2}.

	\centeredfigure{0.5\textwidth}{carro_no_topo_do_morro}

	\begin{answer}
		\measure{44.3}{m/s} ou \measure{159}{km/h}
	\end{answer}

	\begin{solution}
		As duas forças que agem no carro são: a força-peso e a força normal (figura abaixo).

		\centeredfigure{0.5\textwidth}{carro_no_topo_do_morro_solucao}

		Se o veículo estivesse numa pista plana, de modo que a aceleração vertical fosse nula, essas duas forças anulariam-se.
		Porém, esse não é o caso.
		Ao invés disso, o veículo realiza, momentaneamente, movimento circular na direação vertical.
		Para que isso aconteça, é necessária a presença de uma aceleração centrípeta, cujo valor é $v^2/R$, onde $v$ é a velocidade do veículo no topo do morro e $R$, o raio de curvatura da pista naquele ponto.
		Assim, a segunda lei de Newton fica assim:
		\begin{equation*}
			P - N = m\frac{v^2}{R},
		\end{equation*}
		em que consideramos o sentido vertical para baixo como positivo.

		Estamos interessados na situação em que temos a sensação de \emph{ausência} de peso.
		Isso acontece quando a força normal é nula: $N = 0$.
		Então,
		\begin{align*}
			P  &= m\frac{v^2}{R} \\
			mg &= \\
			g  &= \frac{v^2}{R} \\
			gR &= v^2 \\
			\sqrt{gR} &= v,
		\end{align*}
		ou seja, a velocidade para a qual temos a sensação de ausência de peso é dada por $v = \sqrt{gR}$.
		E como $g = \measure{9.81}{m/s^2}$ e $R = \measure{200}{m}$, essa velocidade é:
		\begin{equation*}
			v = \sqrt{9,81 \cdot 200} = \measure{44.3}{m/s}
		\end{equation*}
		ou, se preferir, $v = \measure{159}{km/h}$.
	\end{solution}
\end{question}

\begin{comment}
	```python
	def exercicio_8(r, g = 9.81):
		return(sqrt(g*r))
	```
\end{comment}