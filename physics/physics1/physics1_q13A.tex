\begin{question}
    Uma ginasta de massa \measure{50}{kg} utiliza uma corda presa ao teto para realizar certas acrobacias.
    Em cada situação abaixo, determine a tensão na corda.
    Para isso, suponha que a corda tenha massa desprezível e considere que a aceleração da gravidade é de \measure{10}{m/s^2}.
    \begin{enumerate}
      \item A ginasta sobe pela corda com velocidade constante.
      \item A ginasta desce pela corda com velocidade constante.
      \item A ginasta está suspensa, em repouso.
      \item A ginasta sobe pela corda com aceleração constante de módulo \measure{2}{m/s^2}.
      \item A ginasta desce pela corda com aceleração constante de módulo \measure{3}{m/s^2}.
    \end{enumerate}

    \begin{answer}
      \begin{enumerate}
        \item \measure{500}{N}
        \item \measure{500}{N}
        \item \measure{500}{N}
        \item \measure{600}{N}
        \item \measure{350}{N}
      \end{enumerate}
    \end{answer}

    \begin{solution}
      \begin{enumerate}
        \item Como a velocidade com que a ginasta sobe a corda é constante, a resultante de forças sobre ela é nula (caso contrário observaríamos uma aceleração, conforme a segunda lei de Newton).
        Mas há apenas duas forças atuando: a força-peso, que aponta para baixo, e a tensão na corda, que aponta para cima.
        Então, a intensidade da força peso tem de ser igual à tensão na corda.
        Mas a força peso é dada por $mg = \measure{50}{kg} \cdot \measure{10}{m/s^2} = \measure{500}{N}$.
        Então, essa é a tensão na corda.

        \item O caso é idêntico ao do item anterior: tudo que importa é que a resultante de forças deve ser nula, pois a aceleração observada é nula.

        \item O caso é idêntico ao do item anterior: tudo que importa é que a resultante de forças deve ser nula, pois a aceleração observada é nula.

        \item Nesse caso, a resultante da soma vetorial da tensão $T$ na corda (para cima) com a força-peso $mg$ (para baixo) relaciona-se com a aceleração $a = |\vec a| = \measure{2}{m/s^2}$ (para cima) conforme prescreve a segunda lei de Newton: $T - mg = ma \Rightarrow T = m(g + a) = \measure{600}{N}$.

        \item O caso é análogo ao do item anterior, mas desta vez a resultante aponta para baixo: $mg - T = ma \Rightarrow T = m(g - a) = \measure{350}{N}$, pois agora $a = \measure{3}{m/s^2}$ (para baixo).
        Perceba que, aqui, associamos o sentido ``para baixo'' com o sentido positivo.
        Fizemos isso apenas para escrever o segundo membro da segunda lei de Newton como $ma$ ao invés de $-ma$.
        Mas poderíamos ter mantido o mesmo sistema de coordenadas do item anterior.
        Nesse caso, a segunda lei de Newton ficaria assim: $T - mg = -ma$, que dá o mesmo resultado.
      \end{enumerate}
    \end{solution}
  \end{question}