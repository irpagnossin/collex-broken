% -----------------------------------------------------------------------------------------
% Encoding: UTF-8
% -----------------------------------------------------------------------------------------
% Autor:    Ivan Ramos Pagnossin
% Idioma:   pt-BR
% Licença:  Creative Commons BY-SA International 4.0
% -----------------------------------------------------------------------------------------
\documentclass[a4paper,10pt,oneside]{scrartcl}
	
	\usepackage[utf8]{inputenc}
\usepackage{lmodern}
\usepackage[T1]{fontenc}
\usepackage[brazil]{babel}
\usepackage{tikz}\usetikzlibrary{calc,arrows}

\usepackage{icomma}
\usepackage{amsmath}
\usepackage{amssymb}
\usepackage{xspace}
\usepackage{amsfonts}
\usepackage[locale=FR]{siunitx}
\usepackage{booktabs}
\usepackage{hyperref}
%\usepackage[margin=2.5cm]{geometry}
\usepackage{graphicx}\graphicspath{{physics/assets/}}
\usepackage{icomma}
\usepackage{lastpage}
\usepackage{comment}

\usepackage{irpagnossin-basic}
%\usepackage[answer,solution]{irpagnossin-exam}
\usepackage{irpagnossin-exam}

\usepackage{fancyhdr}
\fancyhead{}
\fancyfoot{}
\lfoot{}
\renewcommand{\headrulewidth}{0pt}
\pagestyle{fancy}

\rfoot{\footnotesize página \thepage\ de \pageref{LastPage}}

% Comandos comuns
\let\emph=\textbf
%\let\vec=\mathbf
%\let\hat=\mathbf
\newcommand\rightanswer{\textcolor{magenta}{$\leftarrow$(resposta correta)}}
\newcommand\vmeasure[2]{\ensuremath{#1}~\unit{#2}}

\newcommand\ava[1]{``\texttt{#1}''}
\newcommand\pd[2]{\ensuremath{\frac{\partial #1}{\partial #2}}}

\newcommand\half{\ensuremath{\frac{1}{2}}}

\NewEnviron{production}[1][]{%
	\color{blue!50!black}
  \vspace{\questionskip}
  \noindent
  \begingroup
    \bfseries\upshape\selectfont
    \color{blue!50!black}
    \MakeUppercase{orientações para produção}%
  \endgroup
  \begingroup
  	\footnotesize
    \ifthenelse{\equal{#1}{}}{}{~(#1)}%
  \endgroup
  \par
  \addpenalty{+300}%
  \footnotesize
  \noindent\BODY
  \par
}{%
  \addpenalty{-300}%
}

\NewEnviron{choices}[1][]{%
  \vspace{\questionskip}
  \color{blue!50!black}
  \noindent
  \begingroup
    \footnotesize
    \bfseries\upshape\selectfont
    \color{blue!50!black}
    \MakeUppercase{alternativas}%
  \endgroup
  \begingroup
    \footnotesize
    \ifthenelse{\equal{#1}{}}{}{~(#1)}%
  \endgroup
  \par
  \addpenalty{+300}%
  \footnotesize
  \noindent\BODY
  \par
}{%
  \addpenalty{-300}%
}


\newcommand\centeredfigure[2]{\begin{center}{\includegraphics[width=#1]{#2}}\end{center}}

\DeclareMathOperator{\sen}{sen}
\let\sin=\sen

	\title{Formulário de Física}
	\author{Prof. Dr. Ivan Ramos Pagnossin}
	\date{\today}

	\lhead{\footnotesize Formulário de Física}
	\rhead{\footnotesize Prof. Dr. Ivan Pagnossin, \today}

\begin{document}

	\begin{gather*}
		\vec F = m\vec a \\
		P = mg, \quad F_\text{atrito} = \mu N, \quad F_\text{centrípeta} = m\frac{v^2}{r},\\
		\vec v = \frac{d\vec r}{dt}, \quad \vec a = \frac{d\vec v}{dt} = \frac{d^2\vec r}{dt^2} \\
		\text{$\vec a$ constante: } \vec v(t) = \vec v_0 + \vec a t, \quad \vec r(t) = \vec r_0 + \vec v_0 t + \frac{1}{2} \vec a t^2 \quad\text{e}\quad
		v^2 = v_0^2 + 2a\Delta x\\
		\text{Energia cinética} = K = \frac{1}{2}mv^2\\
		\text{Energia potencial gravitacional} = U_g = mgh \\
		\text{Energia mecânica} = E = K + U \\
		\text{Momento linear } = p = mv,\\
		\text{Colisão elástica: } p_\text{antes} = p_\text{depois} \quad\text{e}\quad K_\text{antes} = K_\text{depois}\\
		\text{Colisão completamente inelástica: } m_1 v_1 + m_2 v_2 = (m_1 + m_2) v_3 \\
		\text{Torque} = \tau = F_\perp r = F r_\perp = F r \sin\theta \\
		s = \theta r, \quad v = \omega r, \quad a = \alpha r, \quad \omega = 2\pi/T, \quad \text{Frequência} = 1/T\\
		\text{$\alpha$ constante: }\omega(t) = \theta_0 + \alpha t, \quad \theta(t) = \theta_0 + \omega_0 t + \frac{1}{2} \alpha t^2 \quad\text{e}\quad \omega^2 = \omega_0^2 + 2\alpha \Delta\theta\\
		\text{Equilíbrio de translação: } \vec F = \vec 0 \\
		\text{Equilíbrio de rotação: } \vec\tau = \vec 0 \\
		\vec u = u_x\hat i + u_y\hat j \quad\Rightarrow\quad |\vec u| = u = \sqrt{u_x^2 + u_y^2} \\
		\vec u = u\cos\theta \hat i + u\sin\theta \hat j,\\
	\end{gather*}

\end{document}