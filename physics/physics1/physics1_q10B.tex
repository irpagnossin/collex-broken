\begin{question}
	\newcommand\HEIGHT{\measure{3.00}{m}}
	\newcommand\MASS{\measure{5.00}{kg}}
	\newcommand\FORCE{\measure{256}{N}}
	\newcommand\DURATION{\measure{0,150}{s}}
	\newcommand\GRAVITY{\measure{9.81}{m/s^2}}

	A jaqueira é uma árvore trazida da Índia pelos portugueses e cultivada na região amazônica e ao longo da costa brasileira.
	Seu fruto, a jaca, é enorme e pesado: sua massa varia de três a quinze quilogramas e há registros de frutos com massa de até sessenta quilogramas!
	Imagine que uma jaca de \MASS\ solte-se do cacho, a \HEIGHT\ de altura, e atinja o pé do Paulo, que estava ali por perto, talvez tentando apanhar essa jaca.
	\begin{enumerate}
		\item Qual foi a velocidade com que a jaca atingiu o pé do Paulo?
		Explique como chegou a essa conclusão.
		\item Qual foi a variação do momento linear da jaca durante a colisão com o pé do Paulo?
		%\item Explique porque a intensidade média da força exercida pela jaca sobre o pé do Paulo é de \FORCE, sabendo que a colisão durou \DURATION.
	\end{enumerate}

	\begin{answer}
		\begin{enumerate}
			\item \measure{7.67}{m/s}
			\item \measure{38.4}{kg.m/s} ou \measure{-38.4}{kg.m/s} (o sinal não é importante aqui)
			%\item Veja a solução.
		\end{enumerate}
	\end{answer}

	\begin{solution}
		\begin{enumerate}
			\item Durante a queda livre da jaca, a única força sobre ela é a gravitacional.
			Como esta é uma força conservativa, podemos utilizar o princípio da conservação da energia mecânica para determinar a velocidade da jaca imediatamente antes dela atingir o pé do Paulo: no cacho, a jaca tem energia potencial gravitacional $mgh$, mas nenhuma energia cinética, pois ela está parada.
			Por isso, a energia mecânica nessa situação pode ser expressa por $mgh$, onde $m$ é a massa da jaca, $h$ é a altura dela e $g$, a aceleração da gravidade.
			Por outro lado, imediatamente \emph{antes} de atingir o pé do Paulo, a jaca tem enercia cinética $\frac{1}{2}mv^2$, onde $v$ é a velocidade procurada, e nenhuma energia potencial gravitacional.
			E como a energia mecânica a mesma, no cacho e logo antes de atingir o pé do Paulo, então $mgh = \frac{1}{2}mv^2$, o que nos leva a concluir que $v = \sqrt{2gh}$.
			Assim, como $g = \GRAVITY$ e $h = \HEIGHT$, então $v = \sqrt{2 \cdot 9.81 \cdot 3} = \measure{7.67}{m/s}$.

			\item Imediatamente antes de atingir o pé do Paulo, o momento linear da jaca é $mv$ e, após a colisão, é nulo.
			Portanto, a variação é de $\Delta p = 0 - mv = -mv$.
			Assim, como $m = \MASS$ e $v = \measure{7.67}{m/s}$ (item anterior), então $\Delta p = -5 \cdot 7,67 = -\measure{38,4}{kg.m/s}$.

			%\item Como $F = \frac{dp}{dt}$, podemos dizer que $\langle F\rangle = \frac{\Delta p}{\Delta t}$, onde $\Delta p$ é a variação do momento linear durante a colisão, que dura $\Delta t$.
			%Podemos demonstrar esse resultado formalmente da seguinte maneira:
			%\begin{align*}
			%	F &= \frac{dp}{dt} \\
			%	\int_{\tau_0}^{\tau_1} F\,dt       &= \int_{\tau_0}^{\tau_1} \frac{dp}{dt}\,dt \\
			%	                                   &= p(\tau_1) - p(\tau_0) = \Delta p \\
			%	\langle F\rangle (\tau_1 - \tau_0) &= \\
			%	\langle F\rangle \Delta t          &=
			%\end{align*}
			%onde $\tau_0$ e $\tau_1$ são os instantes inicial e final da colisão.
			%Da segunda para a terceira linha, utilizamos o teorema fundamental do Cálculo e, da terceira para a quarta, o teorema do valor médio para integrais.
			%Finalmente, utilizando $\Delta p = \measure{38,4}{kg.m/s}$ (item anterior) e $\Delta t = \DURATION$, concluimos que $\langle F\rangle = 38,4/0,15 = \measure{256}{N}$.
		\end{enumerate}
	\end{solution}
\end{question}

\begin{comment}

```python
from math import sqrt

# Item (a)
def question_10A_a(h, m, dt, g = 9.81):
	return sqrt(2*g*h)

# Item (b)
def question_10A_b(h, m, dt, g = 9.81):
	v = question_10A_a(h, m, dt, g)
	return(-m*v)

# Item (c)
def question_10A_c(h, m, dt, g = 9.81):
	dp = question_10A_b(h, m, dt, g)
	return(dp/dt)
```
\end{comment}