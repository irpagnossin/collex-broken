\begin{question}
	Ao chegar em casa, você nota que o elevador está quebrado e que, por isso, terá de levar as compras pelas escadas.
	\begin{enumerate}
		\item Qual é o trabalho que suas pernas precisam realizar para elevar você, cuja massa é, digamos, \measure{80}{kg}, e as compras, que totalizam \measure{10}{kg}, até o 5\textordmasculine\ andar, aproximadamente \measure{15}{m} acima do nível do andar térreo?
		\item Qual é o trabalho realizado pela força da gravidade nesse mesmo processo?
	\end{enumerate}

	Considere, por simplicidade, que você consiga subir com velocidade constante e que a aceleração da gravidade é de \measure{10}{m/s^2}.

	\begin{answer}
		\begin{enumerate}
			\item \measure{13500}{N}
			\item $-\measure{13500}{N}$
		\end{enumerate}
	\end{answer}

	\begin{solution}
		\begin{enumerate}
			\item Como você sobe com velocidade constante, a intensidade da força $F$ exercida por suas pernas deve ser igual à da força-peso, sobre você e as compras.
			Ou seja, $F = (M + m)g = \measure{900}{N}$, onde $M = \measure{80}{kg}$, $m = \measure{10}{kg}$ e $g = \measure{10}{m/s^2}$.
			E como essa força age ao longo do deslocamento (vertical) $h = \measure{15}{m}$, o trabalho é simplesmente $Fh = \measure{13500}{N}$.
			O trabalho é positivo pois a força exercida por suas pernas é paralela ao deslocamento (vertical).

			\item O deslocamento $h$ é o mesmo e a intensidade da força-peso é igual a $F$ (item anterior).
			A diferença é que a força-peso é anti-paralela ao deslocamento (vertical) $h$.
			Ou seja, o trabalho é simplesmente $-(M + m)gh = -\measure{13500}{N}$.
		\end{enumerate}
	\end{solution}
\end{question}