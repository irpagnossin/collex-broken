\begin{question}
	Lembre-se da sua última viagem pelo terreno acidentado do sul de Minas Gerais.
	Ao passar de carro pelo topo de um pequeno morro, que tem o perfil de uma circunferência de raio $R = \measure{150}{m}$ (figura), você nota uma sensação de perda parcial de peso.
	Isso acontece porque para que o carro acompanhe o perfil da pista (um movimento circular naquele ponto, note), deve haver uma resultante que atue como força centrípeta.
	Como consequência, a força normal, que é a responsável pela nossa sensação de peso, é menor do que seria numa pista plana.
	
	Suponha que você passe por ali a \measure{100}{km/h} e que a massa total do veículo seja de \measure{1.20e3}{kg}.
	Nesse caso, qual é a intensidade da força normal sobre os pneus do seu veículo?
	Considere que a aceleração da gravidade é de \measure{9.81}{m/s^2}.

	\centeredfigure{0.5\textwidth}{carro_no_topo_do_morro}

	\begin{compactdesc}
		\item[Atenção:] note que a velocidade não foi dada em unidades do Sistema Internacional de Unidades.
		\item[Dica:] note que a intensidade da força normal nesse cenário tem de ser inferior à intensidade da força-peso.
	\end{compactdesc}

	\begin{answer}
		\measure{5.59e3}{N}
	\end{answer}

	\begin{solution}
		As duas forças que agem no carro são: a força-peso e a força normal (figura abaixo).

		\centeredfigure{0.5\textwidth}{carro_no_topo_do_morro_solucao}

		Se o veículo estivesse numa pista plana, de modo que a aceleração vertical fosse nula, essas duas forças anulariam-se.
		Porém, esse não é o caso.
		Ao invés disso, o veículo realiza, momentaneamente, movimento circular na direação vertical.
		Para que isso aconteça, é necessária a presença de uma aceleração centrípeta, cujo valor é $v^2/R$, onde $v$ é a velocidade do veículo no topo do morro e $R$, o raio de curvatura da pista naquele ponto.
		Assim, a segunda lei de Newton fica assim:
		\begin{equation*}
			P - N = m\frac{v^2}{R},
		\end{equation*}
		em que consideramos o sentido vertical para baixo como positivo.
		Desse modo, a força normal fica
		\begin{equation*}
			N = P - m\frac{v^2}{R}
			  = mg - m\frac{v^2}{R}
			  = m\left(g - \frac{v^2}{R}\right).
		\end{equation*}
		E como $m = \measure{1.2e3}{kg}$, $g = \measure{9.81}{m/s^2}$, $v = \measure{100}{km/h} = \measure{27.8}{m/s}$ e $R = \measure{150}{m}$, a intensidade da força normal é:
		\begin{equation*}
			N = 1200 \cdot \left(9,81 - \frac{27,8^2}{150}\right) = \measure{5.59e3}{N}.
		\end{equation*}

		Note que tomamos o cuidado de escrever a velocidade $v$ em \unit{m/s}:
		\begin{equation*}
			v = 100 \frac{\unit{km}}{\unit{h}} \cdot \frac{\measure{1000}{m}}{\measure{1}{km}} \cdot \frac{\measure{1}{h}}{\measure{3600}{s}} = \frac{100}{3.6} \unit{m/s} = \measure{27.8}{m/s}.
		\end{equation*}
	\end{solution}
\end{question}

\begin{comment}
	```python
	def exercicio_7(r, v, m, g = 9.81):
		return(m*(g-(v/3.6)**2/r))
	```
\end{comment}