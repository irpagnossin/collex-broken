\section*{Orientações}
	Este curso oferece a você uma série de exercícios para praticar as habilidades inerentes à Mecânica, muitos deles com correções automáticas. Porém, para que isso funcione corretamente, é necessário que você siga as orientações abaixo.

\subsection*{Algarismos significativos}

	De modo geral, os parâmetros numéricos das questões são apresentados com \emph{três algarismos significativos} e espera-se que você forneça suas respostas também com três algarismos significativos.
	Por exemplo, o resultado $1,5672\times10^{-3}$ deve ser expresso como \ava{1.57e-3} ou \ava{0.00157} (sem as aspas).

	Os exercícios têm uma tolerância de 1\% relativamente à resposta esperada.
	Por exemplo, se a resposta correta for $10,0$, o exercício aceita um erro de até $10/100 = 0.1$.
	Então, $9.95$ e $10.06$ são consideradas respostas corretas, mas $9.89$ e $10.101$ não são.

\subsection*{Notação científica}

	Caso precise expressar seus resultados em \emph{notação científica}, faça assim, por exemplo: \ava{1.3e-5} significa ``$1,3\times 10^{-5}$''.

\subsection*{Unidades}

	Este é um curso de Física, de modo que, exceto em raras ocasiões, as unidades fazem parte da resposta.
	Por isso, esteja atento à necessidade deles.

\begin{compactitem}
	\item Informe a unidade logo após o valor numérico, separados por \emph{um} espaço.
	Por exemplo, \measure{1}{m} (um metro) deve ser representado por \ava{1~m} (sem aspas), mas \ava{1m} ou \ava{1~~m} são incorretos.
	
	\item Unidades como o $\Omega$ podem ser representados por seus respectivos nomes.
	Então, \ava{1 ohm} é uma representação correta de \measure{1}{\Omega}.
	
	\item Utilize o ponto (.) para representar o produto de unidades.
	Por exemplo, use \ava{10 N.m} para representar \measure{10}{N.m}.

	\item Utilize a barra (/) para representar a divisão de unidades.
	Por exemplo, use \ava{1 m/s} para representar \measure{1}{m/s}.
	
	\item Utilize o circunflexo para indicar potência de unidades.
	Por exemplo, use \ava{9.81 m/s\^{}2} para representar \measure{9.81}{m/s^2}.
	
	\item A ausência de unidade impõe uma \emph{penalidade} de 10\% do valor do item.
	Por exemplo, se a resposta esperada for \measure{30.2}{m} e você fornecer apenas \ava{30.2} (sem a unidade), obterá apenas 90\% do valor nominal do item.

	\item Salvo quando explicitamente mencionado, espera-se que sua resposta seja expressa no sistema internacional de unidades (SI).
	Porém, os enunciados das questões frequentemente apresentam grandezas em unidades derivadas ou fora do SI.
	Esteja atento a isso.
\end{compactitem}

\subsection*{Grandezas vetoriais}

	Grandezas vetoriais requerem um formato especial: $\vec u = (x,y,z)$, seguido da unidade.
	$x$, $y$ e $z$ são as coordenadas do vetor $\vec u$ na base estabelecida na questão.
	Em geral, essa base será $(\hat i, \hat j, \hat k)$.
	Então, por exemplo, se sua resposta for $\vec u = \vmeasure{(3\hat i + 2\hat k)}{m/s}$, escreva \ava{(3,0,2) m/s} (sem as aspas).

\subsection*{Orientações para produção}

\begin{compactitem}
	\item Todas as questões de múltipla escolha devem ter suas alternativas apresentadas de maneira aleatória para o estudante.
	\item As questões foram ordenadas de modo a dificuldade seja incrementada progressivamente.
	Por isso, é importante que as questões sejam apresentadas na mesma ordem que aqui.
	O mesmo não vale para as questões eleitas como avaliativas.
	Essas podem ser apresentadas em qualquer ordem.
\end{compactitem}