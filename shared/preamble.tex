\usepackage[utf8]{inputenc}
\usepackage{lmodern}
\usepackage[T1]{fontenc}
\usepackage[brazil]{babel}
\usepackage{tikz}\usetikzlibrary{calc,arrows}

\usepackage{icomma}
\usepackage{amsmath}
\usepackage{amssymb}
\usepackage{xspace}
\usepackage{amsfonts}
\usepackage[locale=FR]{siunitx}
\usepackage{booktabs}
\usepackage{hyperref}
%\usepackage[margin=2.5cm]{geometry}
\usepackage{graphicx}\graphicspath{{physics/assets/}}
\usepackage{icomma}
\usepackage{lastpage}
\usepackage{comment}

\usepackage{irpagnossin-basic}
%\usepackage[answer,solution]{irpagnossin-exam}
\usepackage{irpagnossin-exam}

\usepackage{fancyhdr}
\fancyhead{}
\fancyfoot{}
\lfoot{}
\renewcommand{\headrulewidth}{0pt}
\pagestyle{fancy}

\rfoot{\footnotesize página \thepage\ de \pageref{LastPage}}

% Comandos comuns
\let\emph=\textbf
%\let\vec=\mathbf
%\let\hat=\mathbf
\newcommand\rightanswer{\textcolor{magenta}{$\leftarrow$(resposta correta)}}
\newcommand\vmeasure[2]{\ensuremath{#1}~\unit{#2}}

\newcommand\ava[1]{``\texttt{#1}''}
\newcommand\pd[2]{\ensuremath{\frac{\partial #1}{\partial #2}}}

\newcommand\half{\ensuremath{\frac{1}{2}}}

\NewEnviron{production}[1][]{%
	\color{blue!50!black}
  \vspace{\questionskip}
  \noindent
  \begingroup
    \bfseries\upshape\selectfont
    \color{blue!50!black}
    \MakeUppercase{orientações para produção}%
  \endgroup
  \begingroup
  	\footnotesize
    \ifthenelse{\equal{#1}{}}{}{~(#1)}%
  \endgroup
  \par
  \addpenalty{+300}%
  \footnotesize
  \noindent\BODY
  \par
}{%
  \addpenalty{-300}%
}

\NewEnviron{choices}[1][]{%
  \vspace{\questionskip}
  \color{blue!50!black}
  \noindent
  \begingroup
    \footnotesize
    \bfseries\upshape\selectfont
    \color{blue!50!black}
    \MakeUppercase{alternativas}%
  \endgroup
  \begingroup
    \footnotesize
    \ifthenelse{\equal{#1}{}}{}{~(#1)}%
  \endgroup
  \par
  \addpenalty{+300}%
  \footnotesize
  \noindent\BODY
  \par
}{%
  \addpenalty{-300}%
}


\newcommand\centeredfigure[2]{\begin{center}{\includegraphics[width=#1]{#2}}\end{center}}

\DeclareMathOperator{\sen}{sen}
\let\sin=\sen